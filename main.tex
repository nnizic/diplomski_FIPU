\documentclass[a4paper,12pt]{article}
\usepackage[utf8]{inputenc} % za UTF-8 kodiranju
\usepackage{float}          % Za [H] fiksiranje pozicije tablice
\usepackage{array}          % Za p{..} širine stupaca
\usepackage{caption}        % Za \caption
\usepackage{booktabs}       % (opcionalno) ljepše tablice
\usepackage{tikz}           % Za crtanje dijagrama
\usetikzlibrary{positioning, arrows.meta}
\usepackage{hyperref}       % Za \url u referencama
\usepackage{geometry}
\geometry{
 a4paper,
 left=1.25in,
 right=1.25in,
 top=1in,
 }
\usepackage[croatian,english]{babel}    %za hrvatske naslove
\usepackage[nottoc]{tocbibind}  %za pravilan table of content
\usepackage{graphicx}   %za dodavanje slika
\usepackage{amsmath}    %za matematičke forumele
\usepackage{subcaption} %za dvije slike u jednoj
\usepackage{booktabs}   %za bolje tablice
\usepackage{cite}

\begin{document}

\begin{center}
SVEUČILIŠTE JURJA DOBRILE U PULI 

FAKULTET INFORMATIKE

\vspace{45mm} 

\textbf{Ime Prezime}

\vspace{20mm} 

\textbf{Naslov rad}

\vspace{5mm}
DIPLOMSKI/ZAVRSNI RAD

\vfill

%Upisat tocan mjesec i godinu
Pula, rujan, 2025. godine
\end{center}

\pagenumbering{gobble}
\clearpage
\newpage

\begin{center}
SVEUČILIŠTE JURJA DOBRILE U PULI 

FAKULTET INFORMATIKE

\vspace{45mm} 

\textbf{Neven Nižić}

\vspace{20mm} 

\textbf{"Optimizacija raspodjele projektnih aktivnosti primjenom genetskih algoritama i Monte Carlo simulacije"}

\vspace{5mm}
DIPLOMSKI RAD

\end{center}

\vspace{45mm}

\textbf{JMBAG: 0303118917, izvanredni student}

\textbf{Studijski smjer: Informatika}
\bigskip

\textbf{Kolegij: Modeliranje i simulacije}

\textbf{Znanstveno područje : Društvene znanosti}

\textbf{Znanstveno polje : Informacijske i komunikacijske znanosti}

\textbf{Znanstvena grana : Informacijski sustavi i informatologija}
\bigskip

\textbf{Mentor: doc.dr. Darko Etinger}

\vfill

\begin{center}

%Upisat tocan mjesec i godinu
Pula, rujan, 2025. godine

\end{center}

\pagenumbering{gobble}
\clearpage
\newpage

%Resetirat margine
\restoregeometry

\selectlanguage{croatian}
\begin{abstract}
Upravljanje projektima često uključuje složene odluke vezane uz raspodjelu aktivnosti i resursa, osobito u uvjetima nesigurnosti i vremenskih ograničenja. Tradicionalne metode kao što su PERT i CPM često ne uspijevaju obuhvatiti stohastičku prirodu stvarnih projekata. U ovom radu razvijen je i evaluiran hibridni optimizacijski pristup temeljen na genetskim algoritmima (GA) i Monte Carlo (MC) simulaciji. Kroz dvofazni eksperimentalni dizajn, provedena je sustavna usporedba tri modela: nasumične pretrage, jedno-kriterijskog GA usmjerenog isključivo na povrat na investiciju (ROI), te više-kriterijskog GA+MC modela (NSGA-II) koji istovremeno optimizira ROI i rizik trajanja projekta. Rezultati dobiveni na sintetičkim podacima različite složenosti i restriktivnosti pokazuju da, iako jedno-kriterijski GA najučinkovitije maksimizira profit, hibridni GA+MC model uspješno generira Paretov front rješenja koja nude optimalan kompromis između profitabilnosti i trajanja. Nadalje, istraživanje je otkrilo ključan nalaz: pod ekstremno restriktivnim budžetom, robusnost jednostavnijeg, jedno-kriterijskog GA nadmašuje onu složenijeg, više-kriterijskog modela. Rad zaključuje da ne postoji univerzalno superioran model, već da optimalan izbor ovisi o strateškim prioritetima – maksimizaciji profita naspram uravnoteženog upravljanja rizikom.
\end{abstract}
\begin{small}
\textbf{Ključne riječi:} projektno upravljanje, genetski algoritam, Monte Carlo simulacija, više-kriterijska optimizacija, upravljanje rizikom, Paretov front, NSGA-II
\end{small}

\bigskip

\selectlanguage{english}
\begin{abstract}
Project management often involves complex decisions regarding the allocation of activities and resources, especially under conditions of uncertainty and constraints. Traditional methods such as PERT and CPM frequently fail to capture the stochastic nature of real-world projects. This thesis develops and evaluates a hybrid optimization approach based on genetic algorithms (GA) and Monte Carlo (MC) simulation. Through a two-phase experimental design, a systematic comparison of three models was conducted: random search, a single-objective GA focused solely on return on investment (ROI), and a multi-objective GA+MC model (NSGA-II) that simultaneously optimizes ROI and project duration risk. The results, obtained from synthetic data of varying complexity and restrictiveness, show that while the single-objective GA is most effective at maximizing profit, the hybrid GA+MC model successfully generates a Pareto front of solutions offering an optimal trade-off between profitability and duration. Furthermore, the research revealed a key finding: under extremely restrictive budget constraints, the robustness of the simpler, single-objective GA surpasses that of the more complex, multi-objective model. The thesis concludes that there is no universally superior model; rather, the optimal choice depends on strategic priorities—maximizing profit versus balanced risk management.
\end{abstract}
\begin{small}
\textbf{Keywords:} project management, genetic algorithm, Monte Carlo simulation, multi-objective optimization, risk management, Pareto front, NSGA-II
\end{small}

\pagenumbering{gobble}
\clearpage
\newpage

\selectlanguage{croatian}
\tableofcontents
\pagenumbering{gobble}
\clearpage
\newpage

\pagenumbering{arabic}

\section{Uvod}
\label{chap:uvod}


U suvremenom projektnom upravljanju, alokacija ograničenih resursa na niz konkurentskih aktivnosti predstavlja jedan od fundamentalnih izazova. Ovaj problem optimizacije može se elegantno modelirati kroz klasični \textit{problem ruksaka} (\textit{Knapsack Problem}), gdje je cilj odabrati podskup stavki na način koji maksimizira ukupnu vrijednost, a da se pritom ne prekorači zadani kapacitet \cite{Kellerer2004}. U domeni projekata, to se preslikava na odabir portfelja aktivnosti koji će generirati najveći mogući povrat na investiciju (ROI), uz strogo poštivanje proračunskih, vremenskih i drugih resursnih ograničenja.

Iako tradicionalne metode poput PERT-a (\textit{Program Evaluation and Review Technique}) i CPM-a (\textit{Critical Path Method}) \cite{Malcolm1959,Kerzner2017} pružaju nezamjenjiv okvir za planiranje i praćenje, njihova primjena nailazi na poteškoće u dinamičnim okruženjima. One su često determinističke prirode i teško se nose s dva ključna izazova: kombinatornom eksplozijom mogućih rješenja i inherentnom nesigurnošću koja prati procjene trajanja i troškova.

Pregled literature ukazuje na to da su genetski algoritmi (GA) iznimno učinkoviti u rješavanju NP-teških optimizacijskih problema \cite{Holland1975,Goldberg1989}, dok Monte Carlo (MC) simulacije predstavljaju zlatni standard za modeliranje i kvantifikaciju rizika \cite{Metropolis1949,Rubinstein2016}. Unatoč tome, primjetan je nedostatak radova koji sustavno istražuju sinergiju ovih dviju moćnih tehnika. Upravo se u tom prostoru pozicionira ovaj rad, s ciljem popunjavanja uočene praznine kroz razvoj i rigoroznu evaluaciju hibridnog GA-MC modela. Fokus nije samo na kombinaciji, već na stvaranju sustavnog eksperimentalnog okvira za kalibraciju i usporednu analizu takvih modela, kako bi se precizno kvantificirao utjecaj ove sinergije na robusnost i kvalitetu konačnih rješenja.

\subsection{Motivacija i značaj istraživanja}

Motivacija za ovo istraživanje proizlazi iz fundamentalne razlike između projektnog plana i stvarne izvedbe. U praksi, projekti se rijetko odvijaju točno prema početnim predviđanjima \cite{Kerzner2017}. Fluktuacije u dostupnosti resursa, nepredviđeni događaji, izmjene u zahtjevima dionika te inherentna nepreciznost početnih procjena česti su uzroci odstupanja, što klasične metode planiranja čini nedovoljno fleksibilnima \cite{PMI2021}.

Stoga se javlja potreba za pristupima koji nadilaze puku alokaciju resursa i ulaze u domenu strateškog upravljanja nesigurnošću. Potrebne su metode koje ne samo da pronalaze matematički optimalan raspored, već vrednuju rješenja i prema njihovoj otpornosti na nepredviđene okolnosti. Ovdje komplementarna priroda genetskih algoritama i Monte Carlo simulacija dolazi do punog izražaja. Genski algoritmi pružaju mehanizam za efikasno pretraživanje astronomski velikog prostora mogućih portfelja \cite{Goldberg1989}, dok Monte Carlo simulacije služe kao "simulator stvarnosti", testirajući svako potencijalno rješenje u tisućama mogućih budućnosti kako bi se procijenila njegova stvarna varijabilnost i rizik \cite{Rubinstein2016}.

Kombiniranjem ovih metoda, ovaj rad prelazi s tradicionalnog pitanja "Koji je najbolji plan?" na puno značajnije pitanje: "Koji je najrobusniji plan koji nudi najbolji kompromis između profita i rizika?".

\subsection{Izazov nesigurnosti u projektnom upravljanju}

Rizici u projektnom upravljanju obuhvaćaju sve događaje ili uvjete koji, ako se pojave, mogu negativno utjecati na ciljeve projekta \cite{Hillson2009}. Oni mogu biti tehničke, organizacijske, financijske ili tržišne prirode, a često su međusobno povezani \cite{PMI2021}. Nesigurnosti, s druge strane, proizlaze iz nepotpunih informacija, promjenjivih uvjeta i fundamentalne nemogućnosti preciznog predviđanja budućih događaja \cite{Smith2014}. Učinkovito upravljanje projektima stoga zahtijeva sustavan pristup identifikaciji, procjeni i razvoju strategija odgovora na rizike i nesigurnosti \cite{Hillson2009}.

U domeni kvantitativne analize, Monte Carlo simulacije predstavljaju jedan od najmoćnijih pristupa \cite{Vose2008}. Metoda se temelji na generiranju velikog broja mogućih scenarija temeljenih na specificiranim distribucijama vjerojatnosti za neizvjesne ulazne parametre. Analizom distribucije dobivenih ishoda, moguće je s visokom pouzdanošću procijeniti, primjerice, vjerojatnost završetka projekta unutar zadanog roka ili budžeta \cite{Rubinstein2016}.

\subsection{Ciljevi, doprinos i struktura rada}

Primarni cilj ovog diplomskog rada jest razviti i kroz sustavne eksperimente evaluirati hibridni GA-MC model za optimizaciju portfelja projektnih aktivnosti. Svrha modela je pružiti podršku u donošenju odluka koje su informirane ne samo o potencijalnoj dobiti, već i o pripadajućem riziku. Doprinos rada ostvaren je na teorijskoj, metodološkoj i praktičnoj razini. Na teorijskoj razini, rad povezuje koncepte iz evolucijskog računarstva s kvantitativnim metodama upravljanja rizikom u jedinstven okvir primijenjen na projektno upravljanje. S metodološkog stajališta, razvijen je i validiran cjelovit dvo-fazni proces za evaluaciju složenih hibridnih modela, koji uključuje fazu kalibracije i fazu usporedne analize. Konačno, praktični doprinos očituje se u demonstraciji primjenjivosti pristupa na sintetičkim podacima, čime se dobiva uvid u performanse i ograničenja različitih optimizacijskih strategija.

Kako bi se ostvario primarni cilj, istraživanje je vođeno sljedećim ključnim pitanjima, od kojih svako predstavlja zaseban sloj analize:

\begin{enumerate}
\item U kojim uvjetima i prema kojim metrikama (npr. profitabilnost, rizik trajanja, stabilnost) hibridni GA-MC pristup postiže superiorne rezultate u odnosu na samostalnu primjenu GA ili MC modela? \\
Ovo pitanje predstavlja temeljnu usporednu analizu. Njegova svrha nije samo potvrditi superiornost hibridnog modela, već i detaljno kvantificirati tu superiornost kroz višestruke metrike – od maksimalnog postignutog ROI-a, preko procijenjenog trajanja, pa sve do pouzdanosti samih rezultata. Odgovor na ovo pitanje omogućit će nam da jasno razumijemo u kojim situacijama je kombinirani pristup neprikosnoven, a kada bi mogao biti prekomjeran.

\item U kojoj mjeri integracija Monte Carlo simulacije u proces evaluacije može poboljšati robusnost rješenja dobivenih genetskim algoritmom? \\
Ovo pitanje zadire u samu srž sinergije dviju tehnika. Njegov cilj je testirati temeljnu hipotezu da algoritam koji uči iz stohastičkih, a ne samo determinističkih podataka, stvara rješenja koja su otpornija na inherentne nesigurnosti projektnih varijabli. Kroz eksperimente, nastojat ćemo dokazati da se rješenja hibridnog modela, iako možda ne dosežu apsolutno najveći ROI, ponašaju predvidljivije i manje su podložna negativnim iznenađenjima u stvarnim uvjetima.

\item Kakav je utjecaj kombiniranog pristupa na stabilnost i pouzdanost optimizacije u uvjetima neizvjesnosti? \\
Treće pitanje fokusira se na praktičnu vrijednost modela za donositelja odluka. Stabilnost i pouzdanost, mjerene standardnom devijacijom, kritični su parametri koji određuju hoće li se algoritam koristiti kao pouzdan alat ili kao nepouzdana "crna kutija". Odgovor na ovo pitanje otkrit će koliko je hibridni model konzistentan u pronalaženju kvalitetnih rješenja kroz više nezavisnih pokretanja, čime će se potvrditi njegova praktična primjenjivost za strateško planiranje.
\end{enumerate}

Sinergija između genskih algoritama i Monte Carlo simulacija proizlazi iz njihove komplementarne prirode: GA učinkovito pretražuje veliki prostor mogućih rješenja i pronalazi visokokvalitetne kandidate \cite{Goldberg1989}, dok MC kvantificira rizik i procjenjuje varijabilnost tih rješenja kroz stohastičko modeliranje \cite{Rubinstein2016}.

\subsection{Od pristupa do rješenja: put do optimalnih kompromisa}

Metodološki pristup ovog rada nije usmjeren samo na demonstraciju superiornosti jedne tehnike, već na dubinsko razumijevanje njezinih mehanizama i ograničenja. Eksperimentalni dio započinje ablacijskom studijom, gdje će se detaljno analizirati utjecaj ključnih genetskih operatora poput križanja i mutacije. Ta studija će empirijski pokazati da, suprotno intuiciji, strategija širine pretrage (veća populacija) daje značajno bolje i stabilnije rezultate od strategije dubine pretrage (više generacija). Ovaj nalaz poslužit će kao temelj za kalibraciju "šampionske" konfiguracije genetskog algoritma za kasniju usporedbu.

Nakon kalibracije, rad će usporediti tri modela: nasumičnu pretragu, klasični \texttt{GA} fokusiran na \texttt{ROI}, i napredni hibridni \texttt{GA+MC} model. Usporedba će otkriti kako se svaki model ponaša u tri scenarija složenosti i budžetskih ograničenja. Očekuje se da će klasični \texttt{GA} dokazati svoju moć kao iznimno učinkovit "profitni maksimizator", dok će nasumična pretraga poslužiti kao jasan dokaz da je za rješavanje problema realne veličine primjena metaheuristika nužna, a ne samo poželjna.

Srž ovog istraživanja leži u evaluaciji hibridnog \texttt{GA+MC} modela. On je razvijen s ciljem da nadilazi tradicionalni "crna kutija" pristup, koji isporučuje samo jedno, optimalno rješenje. Umjesto toga, hibridni model će generirati textbf{Paretov front} – skup ne-dominiranih rješenja koja vizualno predstavljaju kompromis (\textit{trade-off}) između profitabilnosti (\texttt{ROI}) i rizika (trajanje projekta). Rad će dokazati da je ova vizualizacija ključni alat za strateško odlučivanje, omogućujući donositelju odluka da odabere rješenje koje najbolje odgovara njegovoj toleranciji na rizik.

Konačno, analiza će razotkriti i neočekivan, ali ključan nalaz: složenost ne mora uvijek značiti superiornost. Pokazat će se da u ekstremno restriktivnim scenarijima, gdje su resursi iznimno ograničeni, napredni hibridni model postaje ranjiv i nestabilan, dok jednostavniji, jedno-kriterijski \texttt{GA} pokazuje iznimnu robusnost. Ovaj nalaz naglašava da odabir modela mora biti kontekstualan i ovisiti o specifičnim uvjetima i ograničenjima. S ovim proširenjem, uvod će ne samo predstaviti problem, već i stvoriti temelj za duboko razumijevanje svih ključnih rezultata koji slijede.

Ostatak rada organiziran je kako slijedi. \textbf{Drugo poglavlje} pruža pregled relevantne literature i teorijskih osnova, uključujući problem ruksaka, genetske algoritme i metode modeliranja nesigurnosti. \textbf{Treće poglavlje} detaljno opisuje matematički i konceptualni model problema koji se rješava. \textbf{Četvrto poglavlje} ulazi u detalje implementacije, opisujući arhitekturu razvijenog softverskog sustava i korištene programske biblioteke. \textbf{Peto poglavlje} čini srž rada, prikazujući dvo-fazni eksperimentalni dizajn, analizu dobivenih rezultata i detaljnu diskusiju o performansama testiranih modela. Konačno, \textbf{šesto poglavlje} donosi sintezu cjelokupnog rada, sažima ključne doprinose, osvrće se na ograničenja provedenog istraživanja i nudi preporuke za budući rad.


\newpage

\section{Teorijska podloga}
Ovo poglavlje pruža pregled temeljnih teorijskih koncepata ključnih za razumijevanje predloženog modela optimizacije projektnih aktivnosti. Detaljno će se objasniti Problem ruksaka kao osnova za formulaciju problema raspodjele resursa, Genetski algoritmi kao optimizacijska metaheuristika, te Monte Carlo simulacija i metode procjene trajanja kao alati za modeliranje i analizu nesigurnosti u projektnom upravljanju.

\subsection{Knapsack problem}
Problem ruksaka (engl. Knapsack Problem) jedan je od najpoznatijih i najčešće proučavanih problema kombinatorne optimizacije, svrstan u klasu NP-teških problema \cite{Goldberg1989}. U osnovnoj verziji, cilj je odabrati podskup objekata s pridruženom težinom i vrijednošću, s ciljem maksimiziranja ukupne vrijednosti odabranih objekata, pri čemu njihova ukupna težina ne smije prelaziti zadani kapacitet ruksaka. Formalno, za skup od n objekata, gdje svaki objekt i ima težinu $w_i$ i vrijednost $v_i$, te uz zadani kapacitet ruksaka W, cilj je maksimizirati funkciju:
$$
\max \sum_{i=1}^n v_i x_i \quad \text{uz ograničenje} \quad \sum_{i=1}^n w_i x_i \leq W, x_i \in \{0,1\}
$$
gdje je $x_i=1$ ako je objekt i odabran, a $x_i=0$ ako nije.

U kontekstu upravljanja projektima, ovaj se problem često pojavljuje u složenijim varijantama, poput višedimenzionalnog problema ruksaka (Multi-Dimensional Knapsack Problem – MDKP). U MDKP-u, projektne aktivnosti se mogu interpretirati kao objekti s određenom vrijednošću (npr. strateška važnost, povrat investicije – ROI), dok svaka aktivnost troši više vrsta resursa (npr. vrijeme, budžet, broj radnika), koji predstavljaju različite dimenzije "težine". Kapacitet ruksaka tada predstavlja ukupnu raspoloživost svakog od resursa. MDKP se stoga koristi kao snažan model za optimalnu raspodjelu ograničenih, višestrukih resursa među konkurentskim projektnim aktivnostima.

\subsection{Genetski algoritmi}
Genetski algoritmi (GA) su moćne metaheurističke optimizacijske metode inspirirane procesima prirodne selekcije i evolucije \cite{Goldberg1989, Mitchell1998}. Pripadaju široj klasi evolucijskih algoritama i iznimno su učinkoviti u rješavanju složenih optimizacijskih problema s velikim i nepreglednim prostorom rješenja, posebno onih NP-teških, za koje klasične metode nisu praktične \cite{Gandomi2013, Kaveh2012}.

Osnovni princip GA leži u simulaciji evolucije populacije potencijalnih rješenja. Svako rješenje problema kodira se kao kromosom (u ovom radu, binarni niz), a populacija kromosoma se iterativno poboljšava kroz generacije primjenom genetskih operatora. Tijek genetskog algoritma uključuje:
\begin{enumerate}
    \item \textbf{Inicijalizacija populacije:} Generira se početni skup nasumičnih kromosoma.
    \item \textbf{Evaluacija funkcije cilja (fitness):} Svakom kromosomu dodjeljuje se vrijednost pogodnosti (fitnessa) koja odražava kvalitetu rješenja. U ovisnosti o cilju, funkcija pogodnosti može biti jedno-kriterijska (npr. maksimizacija ROI-a) ili više-kriterijska. U naprednijim modelima, kao što je hibridni model razvijen u ovom radu, fitness funkcija može uključivati i rezultate Monte Carlo simulacije kako bi se procijenila robusnost rješenja.
    \item \textbf{Selekcija roditelja:} Kromosomi s višim fitnessom imaju veću vjerojatnost da budu odabrani kao roditelji za stvaranje sljedeće generacije.
    \item \textbf{Križanje (crossover):} Dva odabrana roditelja kombiniraju se kako bi se stvorili novi potomci, prenoseći genetski materijal i istražujući nove dijelove prostora rješenja.
    \item \textbf{Mutacija:} Slučajne, male promjene unose se u kromosome potomaka kako bi se održala genetska raznolikost populacije i izbjegla prerana konvergencija.
    \item \textbf{Zamjena populacije:} Nova generacija potomaka zamjenjuje dio ili cijelu staru populaciju, i proces se ponavlja dok se ne ispuni kriterij zaustavljanja (npr. zadan broj generacija).
\end{enumerate}

\subsection{Monte Carlo simulacija}
Monte Carlo simulacija (MCS) je računska metoda koja koristi nasumično uzorkovanje za procjenu ponašanja složenog sustava ili procesa, posebno kada je analitičko rješenje teško ili nemoguće. Njena je primarna prednost sposobnost modeliranja nesigurnosti i rizika u sustavima s probabilističkim ulaznim varijablama \cite{Vose2008}. U kontekstu projektnog upravljanja, MCS je vrijedan alat za procjenu vjerojatnih ishoda projekta, poput trajanja i troškova, uzimajući u obzir varijabilnost aktivnosti \cite{Miller2009, Avlijas2008}.

Ključni elementi MCS uključuju:
\begin{itemize}
    \item \textbf{Definiranje slučajnih varijabli:} Identificiraju se ulazne varijable čija je vrijednost neizvjesna (npr. trajanje aktivnosti).
    \item \textbf{Odabir distribucije vjerojatnosti:} Za svaku varijablu odabire se distribucija koja najbolje opisuje njeno ponašanje.
    \item \textbf{Generiranje nasumičnih uzoraka:} Velik broj uzoraka generira se iz odabranih distribucija.
    \item \textbf{Provođenje simulacije:} Za svaki skup uzoraka provodi se izračun modela (npr. zbrajanje trajanja aktivnosti).
    \item \textbf{Analiza rezultata:} Nakon velikog broja iteracija, prikupljeni podaci se analiziraju statistički kako bi se dobila distribucija mogućih ishoda.
\end{itemize}

\subsection{Modeliranje nesigurnosti trajanja: PERT i Trokutasta distribucija}
Metodologija PERT (Program Evaluation and Review Technique) uvela je praksu korištenja tri vremenske procjene za aktivnosti s neizvjesnim trajanjem \cite{Malcolm1959}:
\begin{itemize}
    \item \textbf{$T_o$} – optimistična procjena trajanja (najkraće moguće trajanje).
    \item \textbf{$T_m$} – najvjerojatnija procjena trajanja (očekivano trajanje).
    \item \textbf{$T_p$} – pesimistična procjena trajanja (najduže moguće trajanje).
\end{itemize}
Dok tradicionalna PERT metoda koristi ove tri točke za izračun parametara Beta distribucije, u modernoj praksi upravljanja rizikom, a posebno u Monte Carlo simulacijama, često se koristi Trokutasta distribucija zbog svoje jednostavnosti i intuitivnosti \cite{Law2015}.

Trokutasta distribucija je kontinuirana distribucija vjerojatnosti definirana s tri parametra: minimum ($a$), maksimum ($b$) i najvjerojatnija vrijednost ($c$), što direktno odgovara procjenama $T_o$, $T_p$ i $T_m$. Njena je glavna prednost što ne zahtijeva opsežne povijesne podatke, već se može temeljiti na stručnom iskustvu, što je čini iznimno pogodnom za projektno planiranje. Slučajne vrijednosti generirane iz ove distribucije nalaze se unutar intervala [$T_o$, $T_p$], s najvećom vjerojatnošću pojavljivanja oko vrijednosti $T_m$. Prosječna vrijednost (očekivano trajanje) za Trokutastu distribuciju računa se jednostavnom formulom:
$$
E(T) = \frac{T_o + T_m + T_p}{3}
$$
Upravo je Trokutasta distribucija, zbog navedenih prednosti, odabrana kao temelj za modeliranje nesigurnosti trajanja aktivnosti u Monte Carlo simulacijama provedenim u ovom radu.

\newpage

\section{Model problema}
\label{sec:model_problema}

Projekt optimizacije raspodjele projektnih aktivnosti može se formalno predstaviti kao skup od $n$ zadataka, pri čemu svaki zadatak $i$ ima definirane sljedeće karakteristike:
\begin{itemize}
    \item \textbf{procijenjeno trajanje} $d_i$,
    \item \textbf{trošak} $c_i$,
    \item \textbf{vrijednost} odnosno povrat ulaganja (ROI) $v_i$,
    \item \textbf{distribucija nesigurnosti} koja opisuje varijabilnost trajanja i/ili troška.
\end{itemize}

Ovakav formalizam omogućuje primjenu metoda kombinatorne optimizacije \cite{Glover1986} i stohastičkih simulacija \cite{Law2015} za donošenje optimalnih odluka u uvjetima ograničenih resursa.

\subsection{Ograničenja}
Projekt je podložan realnim ograničenjima resursa, koja se u modelu izražavaju na sljedeći način:
\begin{align}
    \sum_{i=1}^n d_i &\leq T_{\mathrm{max}} \quad \text{(ukupno vrijeme)} \label{eq:time_constraint} \\
    \sum_{i=1}^n c_i &\leq B_{\mathrm{max}} \quad \text{(budžet)} \label{eq:budget_constraint} \\
    \sum_{i=1}^n r_i &\leq R_{\mathrm{max}} \quad \text{(maksimalan broj radnika)} \label{eq:workers_constraint}
\end{align}
gdje $T_{\mathrm{max}}$ označava raspoloživo ukupno vrijeme, $B_{\mathrm{max}}$ raspoloživi budžet, a $R_{\mathrm{max}}$ maksimalan broj raspoloživih radnika.

\subsection{Cilj optimizacije}
Cilj optimizacije je \textbf{maksimizirati ukupnu vrijednost projekta} ostajući unutar svih definiranih ograničenja:
\begin{equation}
    \max \sum_{i=1}^n v_i \cdot x_i
    \label{eq:objective}
\end{equation}
pri čemu $x_i \in \{0,1\}$ označava binarnu varijablu koja označava odabir zadatka $i$ za izvršenje.

\subsection{Grafički prikaz modela}
Na slici \ref{fig:model_problema} prikazan je konceptualni model problema optimizacije raspodjele projektnih aktivnosti, uključujući ulazne podatke, ograničenja i ciljnu funkciju.

\begin{figure}
    \centering
    \includegraphics[width=0.9\textwidth]{slike/model_problema.png}
    \caption{Model problema optimizacije raspodjele projektnih aktivnosti}
    \label{fig:model_problema}
\end{figure}

\subsection{Zaključak}
Ovaj model problema omogućuje matematičku i vizualnu formalizaciju optimizacijskog zadatka. Jasna definicija ulaza, ograničenja i cilja ključna je za primjenu optimizacijskih metoda poput genetskih algoritama \cite{Goldberg1989} u kombinaciji s Monte Carlo simulacijama kako bi se dobila rješenja visoke kvalitete u uvjetima nesigurnosti.

\newpage


\section{Implementacija}

Razvijeni model optimizacije implementiran je u programskom jeziku \textbf{Python (verzija 3.x)}, odabranom zbog čitljivosti, bogatog ekosustava biblioteka i široke primjene u znanstvenom računarstvu \cite{PythonSoftwareFoundation}. Python omogućuje brzu izradu prototipa, jednostavnu integraciju modula te učinkovitu obradu i vizualizaciju podataka.

\subsection{Korištene biblioteke}

Za izradu sustava korištene su sljedeće biblioteke (Tablica~\ref{tab:biblioteke}):

\begin{table}[H]
\centering
\caption{Korištene biblioteke u implementaciji}
\label{tab:biblioteke}
\begin{tabular}{|l|p{10cm}|}
\hline
\textbf{Biblioteka} & \textbf{Namjena i citat} \\ \hline
Python & Osnovni programski jezik za cjelokupnu implementaciju. \cite{PythonSoftwareFoundation} \\ \hline
DEAP & Okvir za razvoj i provedbu evolucijskih algoritama. \cite{DEAP2012} \\ \hline
NumPy & Numeričke operacije i statistička obrada nizova podataka. \cite{Harris2020} \\ \hline
Pandas & Učitavanje, obrada i spremanje tabličnih podataka s rezultatima. \cite{PandasDevelopmentTeam2020} \\ \hline
Seaborn & Kreiranje naprednih statističkih vizualizacija (stupčasti, linijski i raspršeni grafikoni). \cite{Waskom2021} \\ \hline
Matplotlib & Osnovna biblioteka za crtanje na koju se oslanja Seaborn. \cite{Hunter2007} \\ \hline
Random & Standardna Python biblioteka korištena za generiranje slučajnih brojeva i uzorkovanje iz Trokutaste distribucije. \\ \hline
\end{tabular}
\end{table}

\subsection{Struktura sustava}


Sustav razvijen za potrebe ovog rada predstavlja cjeloviti eksperimentalni okvir dizajniran za analizu, kalibraciju i usporedbu optimizacijskih metodologija. Umjesto jednostavnog, monolitnog sustava, arhitektura je modularna i sastoji se od dva glavna analitička modula te jednog pomoćnog modula za obradu rezultata:

\begin{enumerate}
    \item \textbf{Modul za analizu i kalibraciju genetskog algoritma} \\
    Ovaj modul predstavlja temelj istraživanja i odgovara na pitanje: \emph{``Kako optimalno konfigurirati genetski algoritam za rješavanje zadanog problema?''}. Njegova primarna svrha je provođenje detaljne ablacijske studije (Ablation Study) kako bi se ispitao utjecaj svakog ključnog parametra na performanse algoritma.
    
    \textbf{Funkcionalnosti:}
    \begin{itemize}
        \item Sustavno testiranje različitih konfiguracija genetskog algoritma (standardni GA, bez križanja, bez mutacije, s povećanim brojem generacija, s većom populacijom).
        \item Višestruko pokretanje (\( \text{RUNS} = 10 \)) svake konfiguracije radi osiguravanja statističke značajnosti rezultata.
        \item Izračunavanje metrika performansi, uključujući prosječnu vrijednost (mean) i standardnu devijaciju (std) za ROI i procijenjeno trajanje.
        \item \textbf{Izlaz modula:} ``Šampionska'' konfiguracija – skup optimalnih parametara za genetski algoritam koji će se koristiti u daljnjoj analizi.
    \end{itemize}

    \item \textbf{Modul za usporednu analizu optimizacijskih scenarija} \\
    Ovaj modul čini srž diplomskog rada i koristi ``šampionsku'' konfiguraciju, definiranu u prethodnom modulu, za provođenje konačne usporedbe triju različitih pristupa rješavanju problema.
    
    \textbf{Funkcionalnosti:}
    \begin{itemize}
        \item Implementacija i izvršavanje triju ključnih scenarija:
        \begin{itemize}
            \item \textbf{Osnovni model (Random Search):} Slučajan odabir kao temeljna linija za usporedbu.
            \item \textbf{Klasični genetski algoritam:} Optimizacija usmjerena isključivo na maksimizaciju ROI-a.
            \item \textbf{Hibridni GA+MC model (NSGA-II):} Više-objektivna optimizacija koja istovremeno maksimizira ROI i minimizira rizik trajanja procijenjen Monte Carlo simulacijom.
        \end{itemize}
        \item Statistički robusna usporedba temeljem višestrukih pokretanja (\( \text{RUNS} = 10 \)) svakog scenarija.
        \item \textbf{Ulaz modula:} Optimalni parametri genetskog algoritma dobiveni iz Modula~1.
        \item \textbf{Izlaz modula:} Konačna tablica s usporednim rezultatima performansi (ROI, trajanje) i stabilnosti (standardna devijacija) za svaki od triju scenarija.
    \end{itemize}

    \item \textbf{Modul za obradu i vizualizaciju rezultata} \\
    Ovaj modul nije sekvencijalni korak, već pomoćni alat koji služi za interpretaciju rezultata dobivenih iz prva dva modula.
    
    \textbf{Funkcionalnosti:}
    \begin{itemize}
        \item Generiranje preglednih tablica s rezultatima pomoću \texttt{pandas} biblioteke.
        \item Spremanje rezultata u CSV format za daljnju analizu i dokumentaciju.
        \item (Potencijalno) stvaranje grafičkih prikaza, kao što su stupčasti dijagrami za usporedbu prosječnih vrijednosti ili 2D raspršeni dijagrami (scatter plots) za prikaz Paretovog fronta dobivenog iz NSGA-II algoritma.
    \end{itemize}
\end{enumerate}

\begin{figure}[H]
    \centering
    \includegraphics[width=0.9\textwidth]{slike/tijek_istrazivanja.png}
    \caption{Vizualni prikaz toka istraživanja}
    \label{tok istraživanja}
\end{figure}

\subsection{Modeliranje nesigurnosti: Monte Carlo simulacija}

Za svaku projektnu aktivnost definirane su tri točke procjene trajanja:
\[
a \ (\text{optimistična}), \quad
m \ (\text{najvjerojatnija}), \quad
b \ (\text{pesimistična})
\]
Iako u teoriji postoje kompleksnije distribucije poput \textit{Beta-PERT} distribucije, 
za potrebe ovog rada odabrana je \textbf{Trokutasta distribucija (Triangular distribution)} 
zbog svoje praktičnosti, računalne efikasnosti i intuitivnog temelja na tri poznate procjene.

\paragraph{Generiranje trajanja aktivnosti.}
U svakoj iteraciji Monte Carlo simulacije, trajanje svake aktivnosti generira se slučajnom vrijednošću 
unutar raspona $[a, b]$ s najvećom vjerojatnošću u točki $m$.  
Trokutasta distribucija definirana je funkcijom gustoće vjerojatnosti:
\[
f(x) =
\begin{cases}
\frac{2(x-a)}{(b-a)(m-a)}, & a \leq x < m, \\
\frac{2(b-x)}{(b-a)(b-m)}, & m \leq x \leq b, \\
0, & \text{inače}.
\end{cases}
\]

\paragraph{Procjena trajanja portfelja.}
Ukupno trajanje projektnog portfelja u jednoj simulaciji dobiva se zbrojem trajanja svih aktivnosti odabranih u tom portfelju:
\[
T_{\text{portfolio}} = \sum_{i \in S} t_i
\]
gdje je $S$ skup odabranih aktivnosti, a $t_i$ generirano trajanje aktivnosti $i$.

\paragraph{Agregiranje rezultata.}
Monte Carlo simulacija ponavlja se velik broj puta $(\text{NUM\_SIMULATIONS})$, a konačna procjena trajanja portfelja 
dobiva se kao prosječna vrijednost svih simuliranih trajanja:
\[
\overline{T}(S) = \frac{1}{\text{NUM\_SIMULATIONS}} \sum_{k=1}^{\text{NUM\_SIMULATIONS}} T_{\text{portfolio}}^{(k)}
\]
gdje $T_{\text{portfolio}}^{(k)}$ označava ukupno trajanje portfelja u $k$-toj simulaciji.

\subsection{Optimizacijski pristup: Genetski algoritam}

Implementacija genetskog algoritma provedena je pomoću programske biblioteke \texttt{DEAP} (Distributed Evolutionary Algorithms in Python) \cite{DEAP2012}. 
S obzirom na prirodu problema odabira podskupa aktivnosti, korištena je \textbf{binarna reprezentacija}.

\paragraph{Reprezentacija jedinke.}
Svaka jedinka (kromosom) u populaciji predstavlja jedno potencijalno rješenje – jedan portfelj projekata. 
Predstavljena je kao binarni niz duljine jednake ukupnom broju aktivnosti $(\text{NUM\_ACTIVITIES})$, gdje gen na poziciji $i$ ima vrijednost:
\[
g_i =
\begin{cases}
1, & \text{ako je $i$-ta aktivnost odabrana}, \\
0, & \text{ako nije odabrana}.
\end{cases}
\]

\paragraph{Funkcija pogodnosti (Fitness Function).}
Ovisno o eksperimentalnom scenariju, korištene su dvije vrste funkcije pogodnosti:

\begin{enumerate}
    \item \textbf{Jedno-kriterijska optimizacija.}  
Za scenarij GA (samo ROI), cilj je bio isključivo maksimizacija ukupnog povrata na investiciju (ROI). Za rukovanje ograničenjem budžeta primijenjena je stroga kaznena metoda. Ako ukupni trošak odabranog portfelja $S$ ne prelazi budžet, njegova pogodnost je jednaka ukupnom ROI-u. U suprotnom, pogodnost postaje negativna vrijednost proporcionalna iznosu prekoračenja:
$$
\text{Fitness}(S) = 
\begin{cases}
    \sum_{i \in S} \text{ROI}_i, & \text{ako } \sum_{i \in S} \text{Trošak}_i \leq \text{Budžet} \\
    -\left(\sum_{i \in S} \text{Trošak}_i - \text{Budžet}\right), & \text{ako } \sum_{i \in S} \text{Trošak}_i > \text{Budžet}
\end{cases}
$$
Ovakav pristup osigurava da svako valjano rješenje (koje ima pozitivan fitness) uvijek bude ocijenjeno kao bolje od bilo kojeg nevaljanog rješenja (koje ima negativan fitness).    \item \textbf{Više-kriterijska optimizacija.}  
    Za hibridni scenarij \texttt{GA+MC} korišten je napredni algoritam NSGA-II, s ciljem istovremene optimizacije dva suprotstavljena kriterija:
    \begin{enumerate}
        \item maksimizirati ROI,
        \item minimizirati prosječno trajanje projekta, procijenjeno Monte Carlo simulacijom.
    \end{enumerate}
    Formalno:
    \[
    \begin{cases}
    \max f_1(S) = ROI(S) \\
    \min f_2(S) = \overline{T}(S)
    \end{cases}
    \]
    gdje $\overline{T}(S)$ označava prosječno trajanje portfelja $S$.
\end{enumerate}

\paragraph{Genetski operatori.}
Za evoluciju populacije korišteni su sljedeći standardni operatori za binarnu reprezentaciju:

\begin{itemize}
    \item \textbf{Selekcija:}  
    Turnirska selekcija (\texttt{tools.selTournament}) za jedno-kriterijsku optimizaciju,  
    te \texttt{tools.selNSGA2} za više-kriterijsku optimizaciju.
    
    \item \textbf{Križanje:}  
    Križanje u dvije točke (\texttt{tools.cxTwoPoint}), koje razmjenjuje segmente između dva roditeljska kromosoma.
    
    \item \textbf{Mutacija:}  
    Slučajna promjena bita (\texttt{tools.mutFlipBit}), koja s malom vjerojatnošću mijenja vrijednost pojedinog gena (iz $0$ u $1$ ili obrnuto), osiguravajući genetsku raznolikost i sprječavajući preranu konvergenciju.
\end{itemize}


\subsection{Vizualizacija}

Za analizu i prikaz rezultata dobivenih optimizacijom korištene su biblioteke \texttt{pandas} za tabličnu obradu podataka te \texttt{}Seaborn} i \texttt{}Matplotlib} \cite{Waskom2021, Hunter2007} za grafičku vizualizaciju.
Kombinacija ovih alata omogućila je jasnu i preglednu prezentaciju rezultata 
dobivenih iz eksperimentalnih scenarija.

Ključni vizualni elementi korišteni u ovom radu uključuju:

\begin{itemize}
    \item \textbf{Tablični prikazi:} Detaljne tablice s konačnim, statistički obrađenim rezultatima 
    usporedbe različitih optimizacijskih scenarija, uključujući osnovne metrike 
    poput prosječnog ROI-a, prosječnog trajanja te raspona vrijednosti.

    \item \textbf{Stupčasti dijagrami:} Koristili su se za vizualnu usporedbu prosječnih vrijednosti 
    (\textit{ROI} i trajanje) između različitih metodologija optimizacije, omogućujući brzu identifikaciju 
    učinkovitijih pristupa.

    \item \textbf{Raspršeni dijagram (Scatter Plot):} Prikaz Paretovog fronta dobivenog NSGA-II algoritmom, 
    koji jasno ilustrira kompromis (\textit{trade-off}) između dvaju suprotstavljenih ciljeva:
    maksimizacije ROI-a i minimizacije trajanja. Time se omogućuje intuitivna procjena 
    učinkovitosti rješenja.
\end{itemize}

Vizualizacija rezultata odigrala je ključnu ulogu u interpretaciji dobivenih podataka, 
posebno u scenarijima s više ciljeva, gdje tablični prikazi sami po sebi nisu dovoljni 
za uočavanje odnosa i kompromisa među varijablama.


\newpage


\section{Eksperimenti i analiza rezultata}

U ovom poglavlju detaljno se opisuje eksperimentalni postav, provedba eksperimenata te analiza i interpretacija dobivenih rezultata. Cilj je bio empirijski validirati predloženi hibridni model i usporediti ga s drugim pristupima.

\subsection{Postavke okruženja i testni podaci}

Svi eksperimenti provedeni su u programskom okruženju Python (verzija 3.x) na standardnom osobnom računalu. Za potrebe istraživanja generiran je sintetički skup podataka koji oponaša realističan projektni portfelj. Skup se sastoji od 50 jedinstvenih projektnih aktivnosti (\texttt{NUM\_ACTIVITIES = 50}). Za svaku aktivnost definirani su sljedeći parametri unutar zadanih raspona:

\begin{itemize}
  \item \textbf{Trošak (cost):} Slučajna cjelobrojna vrijednost između 50 i 200.
  \item \textbf{ROI (roi):} Slučajna decimalna vrijednost između 1.0 i 3.0.
  \item \textbf{Procjene trajanja:}
  \begin{itemize}
    \item Optimistično: između 5 i 10 dana.
    \item Najvjerojatnije: između 10 i 20 dana.
    \item Pesimistično: između 20 i 40 dana.
  \end{itemize}
\end{itemize}

Ukupni raspoloživi budžet za portfelj postavljen je na 1000 jedinica (\texttt{BUDGET = 1000}).

\subsection{Eksperimentalni dizajn}

Kako bi se osigurala metodološka ispravnost i izbjegli proizvoljni zaključci, istraživanje je provedeno kroz dvofazni eksperimentalni proces:

\begin{itemize}
  \item \textbf{Faza 1:} Analiza i kalibracija genetskog algoritma. U prvoj fazi provedena je detaljna ablacijska studija kako bi se utvrdilo koji parametri genetskog algoritma daju najkvalitetnija i najstabilnija rješenja za zadani tip problema. Cilj je bio pronaći ``šampionsku'' konfiguraciju GA.
  
  \item \textbf{Faza 2:} Usporedna analiza optimizacijskih modela. U drugoj fazi, ``šampionska'' konfiguracija GA, dobivena u prvoj fazi, korištena je za provođenje konačne usporedbe triju različitih optimizacijskih scenarija i evaluaciju glavne hipoteze rada.
\end{itemize}

\subsection{Eksperiment 1: Analiza parametara i kalibracija genetskog algoritma}

\textbf{Cilj:} Empirijski provjeriti utjecaj osnovnih genetskih operatora i parametara na performanse algoritma te odabrati optimalnu konfiguraciju za daljnje testiranje.

\textbf{Metodologija:} Provedena je ablacijska studija s pet različitih konfiguracija, gdje je svaka pokrenuta 10 puta (\texttt{RUNS = 10}) radi statističke pouzdanosti. Testirane konfiguracije su bile: \emph{Standardni GA}, \emph{Bez mutacije}, \emph{Bez križanja}, \emph{Više generacija} i \emph{Veća populacija}.

\textbf{Rezultati i diskusija:} Rezultati ablacijske studije prikazani su u Tablici~\ref{tab:ga_ablation} te pružaju uvid u dinamiku ponašanja genetskog algoritma.

\begin{table}[H]
\centering
\caption{Rezultati ablacijske studije za parametre GA}
\label{tab:ga_ablation}
\begin{tabular}{|l|c|c|c|c|}
\hline
\textbf{Postavka} & \textbf{ROI\_mean} & \textbf{ROI\_std} & \textbf{Trajanje\_mean} & \textbf{Trajanje\_std} \\
\hline
Standardni GA     & 28.985  & 1.543  & 199.216  & 10.691 \\
Bez mutacije      & 27.627  & 1.581  & 193.497  & 11.364 \\
Bez križanja      & 25.884  & 1.865  & 191.514  & 9.174  \\
Više generacija   & 31.183  & 0.928  & 205.026  & 13.649 \\
Veća populacija   & \textbf{31.683}  & \textbf{0.720}  & \textbf{213.694}  & \textbf{5.574}  \\
\hline
\end{tabular}
\end{table}

Analiza rezultata potvrđuje obje početne hipoteze. 

Prvo, vidljivo je da su genetski operatori križanje i mutacija esencijalni. Uklanjanje križanja drastično smanjuje performanse (\texttt{ROI\_mean} pada na 25.88), što ukazuje da je rekombinacija dobrih rješenja ključan mehanizam pretrage. Uklanjanje mutacije također smanjuje performanse, potvrđujući njezinu ulogu u održavanju genetske raznolikosti i izbjegavanju prerane konvergencije.

Drugo, povećanje računalnih resursa ima direktan pozitivan utjecaj. I \emph{Više generacija} i \emph{Veća populacija} značajno su nadmašile standardnu konfiguraciju. Konfiguracija \emph{Veća populacija} pokazala se superiornom, ostvarivši najviši prosječni ROI (31.683) uz najnižu standardnu devijaciju (0.720). To ukazuje da za ovaj problem veća početna raznolikost rješenja (širina pretrage) donosi bolje rezultate od dužeg trajanja evolucije (dubina pretrage).

Zanimljivo je primijetiti da konfiguracije s najvišim ROI-em ujedno rezultiraju i najdužim prosječnim trajanjem projekta. To sugerira da su najprofitabilnije aktivnosti inherentno povezane s većim vremenskim ulaganjem, što stvara prirodni kompromis (\emph{trade-off}) između profita i rizika trajanja. Upravljanje tim kompromisom bit će predmet analize u sljedećem eksperimentu.


\begin{figure}[H]
    \centering
    \begin{subfigure}[b]{0.48\textwidth}
        \centering
        \includegraphics[width=\textwidth]{slike/ga_usporedba_roi.png}
        \caption{Usporedba prosječnog ROI-a za različite konfiguracije GA.}
        \label{fig:ga_roi}
    \end{subfigure}
    \hfill
    \begin{subfigure}[b]{0.48\textwidth}
        \centering
        \includegraphics[width=\textwidth]{slike/ga_usporedba_trajanje.png}
        \caption{Usporedba prosječnog trajanja projekta za različite konfiguracije GA.}
        \label{fig:ga_trajanje}
    \end{subfigure}
    \caption{Grafički prikaz rezultata ablacijske studije za genetski algoritam.}
    \label{fig:ga_ablation}
\end{figure}

\textbf{Zaključak Eksperimenta 1:}  
Na temelju empirijskih rezultata, konfiguracija \emph{Veća populacija} odabrana je kao ``šampionska''. Njezini parametri (\texttt{POP\_SIZE = 200}, \texttt{NGEN = 40}, \texttt{CX\_PB = 0.7}, \texttt{MUT\_PB = 0.2}) koristit će se u svim daljnjim eksperimentima koji uključuju genetski algoritam, kako bi se osigurala njihova maksimalna učinkovitost i omogućila pravedna usporedba.

\subsection{Eksperiment 2: Usporedna analiza optimizacijskih modela}

Na slici \ref{fig:tok_eksperimenta} prikazan je dijagram toka izvođenja eksperimenta. 
\begin{figure}[]
    \centering
    \includegraphics[width=0.75\textwidth]{slike/tok_eksperimenta2.png}
    \caption{Dijagram toka izvođenja eksperimenta}
    \label{fig:tok_eksperimenta}
\end{figure}

Nakon što je u Eksperimentu 1 provedena kalibracija i odabrana "šampionska" konfiguracija genetskog algoritma, u drugoj fazi istraživanja pristupilo se ključnoj usporednoj analizi triju razvijenih modela.

\subsubsection{Metodologija}
Cilj ovog eksperimenta bio je kvantitativno i kvalitativno usporediti performanse, kvalitetu i stabilnost rješenja dobivenih pomoću tri različita optimizacijska scenarija. Eksperimenti su provedeni prema planu definiranom u Tablici \ref{tab:plan_eksperimenata}. Svaka konfiguracija iz plana testirana je 10 puta (RUNS=10) radi osiguravanja statističke robusnosti zaključaka. Genetski algoritmi (GA (samo ROI) i GA+MC (NSGA-II)) koristili su "šampionsku" konfiguraciju parametara (POP\_SIZE = 200, NGEN = 40, itd.) utvrđenu u prethodnom koraku, uz skaliranje parametra NGEN sukladno složenosti problema.

\begin{table}[h!]
    \centering
    \resizebox{\textwidth}{!}{
    \begin{tabular}{|l|l|l|l|l|}
        \hline
        \textbf{Eksperiment} & \textbf{NUM\_ACTIVITIES} & \textbf{BUDGET} & \textbf{Pripada seriji} & \textbf{Napomena} \\
        \hline
        A1 & 10 & 1000 & A & Osnovna složenost \\
        \hline
        A2 / B2 & 50 & 2500 & A, B & Centralni / Referentni eksperiment \\
        \hline
        A3 & 100 & 5000 & A & Visoka složenost \\
        \hline
        B1 & 50 & 1500 & B & Restriktivan budžet \\
        \hline
        B3 & 50 & 4000 & B & Labav budžet \\
        \hline
    \end{tabular}
}
    \caption{Plan naprednih eksperimenata}
    \label{tab:plan_eksperimenata}
\end{table}

\subsubsection{Rezultati}
Svi rezultati dobiveni provođenjem Eksperimenta 2 sažeti su u Tablici \ref{tab:rezultati}. Ova tablica predstavlja temelj za daljnju diskusiju i donošenje zaključaka.

\begin{table}[h!]
    \centering
    \resizebox{\textwidth}{!}{
    \begin{tabular}{|l|l|l|l|l|l|}
        \hline
        \textbf{Eksperiment} & \textbf{Scenarij} & \textbf{ROI\_mean} & \textbf{ROI\_std} & \textbf{Trajanje\_mean} & \textbf{Trajanje\_std} \\
        \hline
        A1\_Osnovni & Random Search (MC) & 17.140 & 3.55e-15 & 144.15 & 10.911 \\
        \hline
        A1\_Osnovni & GA (samo ROI) & 17.140 & 3.55e-15 & 143.10 & 1.239 \\
        \hline
        A1\_Osnovni & GA+MC (NSGA-II) & 17.140 & 3.55e-15 & 142.00 & 0.372 \\
        \hline
        A2\_Srednji & Random Search (MC) & 40.108 & 0.703 & 341.02 & 9.405 \\
        \hline
        A2\_Srednji & GA (samo ROI) & 46.125 & 0.412 & 363.63 & 7.843 \\
        \hline
        A2\_Srednji & GA+MC (NSGA-II) & 44.099 & 0.980 & 319.21 & 13.171 \\
        \hline
        A3\_Slozeni & Random Search (MC) & 95.835 & 1.468 & 715.60 & 10.451 \\
        \hline
        A3\_Slozeni & GA (samo ROI) & 114.224 & 0.891 & 792.30 & 11.933 \\
        \hline
        A3\_Slozeni & GA+MC (NSGA-II) & 109.095 & 2.008 & 681.56 & 21.733 \\
        \hline
        B1\_Restriktivan & Random Search (MC) & 24.120 & 1.846 & 197.62 & 20.181 \\
        \hline
        B1\_Restriktivan & GA (samo ROI) & 37.976 & 0.567 & 253.56 & 12.386 \\
        \hline
        B1\_Restriktivan & GA+MC (NSGA-II) & 17.711 & 17.728 & 50104.38 & 49894.619 \\
        \hline
        B3\_Labav & Random Search (MC) & 71.379 & 1.114 & 536.90 & 16.247 \\
        \hline
        B3\_Labav & GA (samo ROI) & 79.065 & 0.518 & 562.19 & 9.345 \\
        \hline
        B3\_Labav & GA+MC (NSGA-II) & 76.949 & 0.487 & 526.61 & 10.602 \\
        \hline
    \end{tabular}
}
    \caption{Konačni rezultati usporedne analize optimizacijskih modela}
    \label{tab:rezultati}
\end{table}

\subsubsection{Diskusija rezultata}
Detaljna analiza rezultata provedena je kroz tri tematske cjeline, uz oslanjanje na vizualizacije generirane iz podataka u Tablici \ref{tab:rezultati}.

\textbf{Analiza Skalabilnosti (Serija A)}
\begin{figure}[H]
    \centering
    \begin{subfigure}[b]{0.48\textwidth}
        \centering
        \includegraphics[width=\textwidth]{slike/grafikoni_final/A_skalabilnost_roi.png}
        \caption{Usporedba prosječnog ROI-a za tri različite konfiguracije.}
        \label{fig:a_ska_roi}
    \end{subfigure}
    \hfill
    \begin{subfigure}[b]{0.48\textwidth}
        \centering
        \includegraphics[width=\textwidth]{slike/grafikoni_final/A_skalabilnost_trajanje.png}
        \caption{Usporedba prosječnog trajanja projekta za tri različite konfiguracije.}
        \label{fig:a_ska_trajanje}
    \end{subfigure}
    \caption{Grafički prikaz rezultata usporednih  studija.}
    \label{fig:a_skalabilnost}
\end{figure}

Kao što je vidljivo na grafikonima \ref{fig:a_ska_roi} i \ref{fig:a_ska_trajanje}, porast složenosti problema s 10 na 100 aktivnosti drastično utječe na performanse modela. Dok su na osnovnom problemu (A1) sve metode pronašle isti financijski optimum, jaz u ROI\_mean vrijednostima eksponencijalno raste u korist genetskih algoritama. U eksperimentu A3, GA (samo ROI) ostvaruje prosječni ROI za preko 18 bodova viši od Random Search metode, što nedvojbeno potvrđuje hipotezu o nužnosti inteligentne pretrage (H1).
Istovremeno, analiza trajanja otkriva postojanje kompromisa. Hibridni model GA+MC (NSGA-II) konzistentno identificira rješenja sa značajno nižim prosječnim trajanjem. Na složenom problemu A3, ta razlika iznosi preko 110 dana u usporedbi s GA (samo ROI). Ovo potvrđuje hipotezu H2 – hibridni model uspješno upravlja rizikom, ali uz mjerljivu "cijenu" u vidu nešto nižeg maksimalnog ROI-a.

\textbf{Analiza Utjecaja Ograničenja (Serija B)}
\begin{figure}[]
    \centering
    \includegraphics[width=0.75\textwidth]{slike/grafikoni_final/B_budzet_roi.png}
    \caption{Budžet}
    \label{fig:budzet_roi}
\end{figure}

Grafikon \ref{fig:budzet_roi} ilustrira ponašanje modela pod različitim proračunskim pritiskom. Najvažniji nalaz dolazi iz eksperimenta s restriktivnim budžetom (B1). U tim uvjetima, GA+MC (NSGA-II) pokazuje iznimnu krhkost, ne uspijevajući pronaći valjano rješenje u 50\% pokretanja, što rezultira katastrofalnim prosječnim performansama (Tablica \ref{tab:rezultati}, redak 11). S druge strane, jednostavniji GA (samo ROI) pokazuje se vrlo robusnim, uspješno pronalazeći visokoprofitabilna rješenja čak i u vrlo ograničenom prostoru. Ovo ukazuje da složenost više-kriterijske pretrage može biti nedostatak u ekstremno suženim prostorima rješenja.
U uvjetima labavog budžeta (B3), svi modeli rade očekivano dobro, a razlike među njima se smanjuju, potvrđujući hipotezu H3.

\textbf{Analiza Stabilnosti i Pouzdanosti}
\begin{figure}[H]
    \centering
    \begin{subfigure}[b]{0.48\textwidth}
        \centering
        \includegraphics[width=\textwidth]{slike/grafikoni_final/C_stabilnost_roi.png}
        \caption{Stabilnost ROI-a za tri različite konfiguracije.}
        \label{fig:stab_roi}
    \end{subfigure}
    \hfill
    \begin{subfigure}[b]{0.48\textwidth}
        \centering
        \includegraphics[width=\textwidth]{slike/grafikoni_final/C_stabilnost_trajanje.png}
        \caption{Stabilnost trajanja projekta za tri različite konfiguracije.}
        \label{fig:stab_trajanje}
    \end{subfigure}
    \caption{Grafički prikaz rezultata usporednih  studija.}
    \label{fig:a_skalabilnost}
\end{figure}

Grafikoni \ref{fig:stab_roi} i \ref{fig:stab_trajanje} prikazuju standardnu devijaciju kao mjeru konzistentnosti. Izvan scenarija B1 gdje je doživio neuspjeh, GA+MC (NSGA-II) model pokazuje usporedivu ili nižu devijaciju trajanja u odnosu na klasični GA. To implicira da rješenja koja nudi nisu samo u prosjeku brža, već su i pouzdanija, odnosno njihovo procijenjeno trajanje manje varira. Ova predvidljivost je od iznimne važnosti za praktično upravljanje projektima.

\subsubsection{Sinteza glavnih zaključaka eksperimenata}
\begin{itemize}
    \item \textbf{Random Search (MC):} Koristan kao početna točka i za jednostavne probleme, ali potpuno neadekvatan kao ozbiljan optimizacijski alat za probleme realne veličine i složenosti.
    \item \textbf{GA (samo ROI):} Izuzetno snažan i robustan "profitni maksimizator". Najbolji je izbor u situacijama gdje je financijska dobit jedini i isključivi kriterij, te pokazuje veliku otpornost u uvjetima strogih ograničenja.
    \item \textbf{GA+MC (NSGA-II):} Sofisticirani "upravitelj rizikom". Njegova najveća vrijednost je u pružanju strateških opcija koje balansiraju profit i rizik (trajanje). Superioran je u standardnim i složenim uvjetima, ali njegova složenost ga čini osjetljivim i nepouzdanim u okruženjima s ekstremno restriktivnim ograničenjima.
\end{itemize}
Konačan izbor modela stoga ovisi o strateškim prioritetima projektnog ureda. Za maksimalan profit, klasični GA je pobjednik. Za uravnoteženo i rizikom informirano donošenje odluka, hibridni GA+MC je superioran, uz nužan oprez pri primjeni u vrlo ograničenim uvjetima.

\newpage

\section{Zaključak}

U ovom diplomskom radu predstavili smo kompleksan pristup optimizaciji raspodjele projektnih aktivnosti koristeći kombinaciju genetskih algoritama i Monte Carlo simulacije. Cilj je bio razviti model koji uzima u obzir nesigurnost u trajanju, troškovima i vrijednosti zadataka, te u okviru zadanih ograničenja vremena, budžeta i raspoloživih resursa maksimizira ukupnu vrijednost projekta.

Kroz detaljnu analizu problematike i pregled postojeće literature, identificirali smo ključne izazove u upravljanju projektima, posebice u segmentu neizvjesnosti i složenosti optimizacije. Implementacijom metaheurističkih metoda, u ovom slučaju genetskih algoritama \cite{Goldberg1989, Mitchell1998}, omogućili smo efikasno pretraživanje velikog prostora rješenja, dok je Monte Carlo simulacija služila za kvantitativno modeliranje rizika i nesigurnosti \cite{Miller2009, Avlijas2008}, dajući time realističniju procjenu performansi optimizacijskog rješenja.

Praktična implementacija rezultirala je modelom koji omogućuje donošenje informiranih odluka u planiranju i upravljanju projektima, pružajući projektnim menadžerima alate za bolje usklađivanje ciljeva i ograničenja. Pokazalo se da je kombinacija ovih metoda učinkovita u pronalasku balansiranih rješenja koja maksimiziraju povrat ulaganja, uz minimizaciju rizika od prekoračenja budžeta ili rokova.

Iako su postignuti rezultati zadovoljavajući, postoje brojna područja za buduća istraživanja i unaprjeđenja, među kojima izdvajamo:
\begin{itemize}
    \item Proširenje modela na dinamičke uvjete projekata koji se mijenjaju tijekom vremena, uključujući nepredvidive vanjske utjecaje.
    \item Integracija dodatnih metaheurističkih i hibridnih algoritama, poput algoritama rojčaste inteligencije ili simuliranog kaljenja, radi poboljšanja kvalitete rješenja \cite{Gandomi2013}.
    \item Primjena tehnika strojnog učenja za preciznije predviđanje distribucija nesigurnosti i automatsku adaptaciju parametara optimizacije.
    \item Razvoj softverskih alata s intuitivnim korisničkim sučeljem za praktičnu primjenu predloženih metoda u realnim projektnim okruženjima.
\end{itemize}

Zaključno, ovaj rad potvrđuje važnost primjene naprednih algoritamskih rješenja u upravljanju projektima, posebno u uvjetima nesigurnosti, te doprinosi boljem razumijevanju i praktičnoj primjeni optimizacijskih i simulacijskih metoda u području projektne ekonomike i menadžmenta. Kao što ističe Kerzner \cite{Kerzner2017}, učinkovito upravljanje projektima u suvremenom okruženju zahtijeva kombinaciju tradicionalnih i naprednih pristupa, a naš model pruža značajan doprinos u tom smjeru.



\newpage

\bibliographystyle{unsrt}
\bibliography{literatura}
\newpage

%Automatski generira listu svih slika
\listoffigures
\newpage

%Automatski generira listu svih tablica
\listoftables
\newpage


\end{document}
