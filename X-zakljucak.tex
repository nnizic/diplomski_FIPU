\section{Diskusija i zaključak}
\label{chap:zakljucak}

Ovaj diplomski rad bavio se složenim problemom optimizacije portfelja projektnih aktivnosti u uvjetima nesigurnosti. S ciljem razvoja modela koji donositeljima odluka nudi ne samo profitabilna, već i robusna rješenja, razvijen je i evaluiran hibridni pristup koji integrira snagu genetskih algoritama za pretraživanje složenih prostora rješenja i Monte Carlo simulacije za kvantifikaciju rizika. Kroz sustavni, dvo-fazni eksperimentalni proces, provedena je prvo kalibracija parametara genetskog algoritma, a zatim i detaljna usporedna analiza triju optimizacijskih modela: osnovne metode nasumične pretrage, klasičnog genetskog algoritma usmjerenog isključivo na povrat na investiciju (ROI), te naprednog, više-kriterijskog hibridnog modela (GA+MC).

\subsection{Glavni nalazi i odgovori na istraživačka pitanja}

Provedeni eksperimenti pružili su jasne i empirijski utemeljene odgovore na istraživačka pitanja postavljena u uvodu rada. Prvo, potvrđena je superiornost inteligentne pretrage. Dok je nasumična pretraga bila donekle uspješna na jednostavnim problemima, njena učinkovitost drastično opada s porastom složenosti, čime je potvrđena hipoteza H1. Genetski algoritmi su se pokazali značajno superiornijima u pronalaženju rješenja s visokim ROI-em.

Drugo, dokazano je da hibridni model uspješno upravlja kompromisom između profita i rizika. Hibridni GA+MC model, temeljen na NSGA-II algoritmu, uspješno je identificirao Paretov front optimalnih rješenja. Time je potvrđena hipoteza H2, jer model donositelju odluke ne nudi jedno, već čitav spektar strateških opcija koje balansiraju viši ROI s dužim procijenjenim trajanjem, i obrnuto. Vizualizacija Paretovog fronta pokazala se kao ključan alat za strateško odlučivanje.

Treće, utvrđeno je da stabilnost i robusnost modela ovise o kontekstu problema. Analiza je otkrila ključan, nijansiran nalaz kojim je potvrđena hipoteza H3. U standardnim uvjetima, hibridni GA+MC model nudi rješenja čije je procijenjeno trajanje pouzdanije. Međutim, pod ekstremnim pritiskom vrlo restriktivnog budžeta, njegova složenost postaje nedostatak, što dovodi do nestabilnosti. U tim uvjetima, jednostavniji, jedno-kriterijski GA pokazao se robusnijim.

\subsection{Doprinos i ograničenja rada}

Doprinos ovog rada ostvaren je na metodološkoj i praktičnoj razini. S metodološkog stajališta, razvijen je i primijenjen cjelovit, dvo-fazni eksperimentalni okvir za evaluaciju optimizacijskih algoritama, koji uključuje fazu kalibracije i fazu usporedne analize. Ključni doprinos je empirijska demonstracija da ne postoji univerzalno "najbolji" model, već da optimalan izbor ovisi o strateškim prioritetima – maksimizaciji profita ili uravnoteženom upravljanju rizikom. Praktični doprinos očituje se u demonstraciji kako sinergija genetskih algoritama i Monte Carlo simulacije može pružiti konkretnu vrijednost projektnim menadžerima. Paretov front, kao ključni rezultat hibridnog modela, transformira optimizacijski problem iz potrage za jednim rješenjem u moćan alat za strateško donošenje odluka.

U tijeku istraživanja uočena su i određena ograničenja koja definiraju domete ovog rada. Najznačajnije ograničenje je računalna zahtjevnost hibridnog GA+MC modela, koji je zbog višestrukih simulacija znatno sporiji od klasičnog GA. Drugo ograničenje je korištenje sintetičkih podataka; iako su parametri odabrani da budu realistični, validacija modela na stvarnim projektnim podacima predstavljala bi važan sljedeći korak. Konačno, pokazalo se da stabilnost pod ograničenjima može biti izazov, jer je napredni više-kriterijski model pokazao krhkost u uvjetima ekstremno restriktivnih resursa.

\subsection{Preporuke za budući rad i završna riječ}

Na temelju provedenog istraživanja i uočenih ograničenja, izdvaja se nekoliko pravaca za budući rad. Jedan od smjerova je hibridizacija s drugim metaheuristikama, poput algoritama rojeva čestica (PSO) ili simuliranog kaljenja, radi potencijalnog poboljšanja brzine konvergencije \cite{Gandomi2013}. Drugi zanimljiv pravac je primjena strojnog učenja za preciznije predviđanje distribucija nesigurnosti iz povijesnih podataka, čime bi se smanjilo oslanjanje na subjektivne procjene. Konačno, praktična primjenjivost modela značajno bi se povećala razvojem softverskog alata s intuitivnim korisničkim sučeljem koje bi omogućilo interaktivnu vizualizaciju Paretovog fronta.

Zaključno, ovaj rad je potvrdio da primjena naprednih algoritamskih rješenja nudi značajan potencijal za unapređenje procesa upravljanja projektima u uvjetima nesigurnosti. Učinkovito upravljanje, kako ističe Kerzner \cite{Kerzner2017}, u suvremenom okruženju zahtijeva upravo kombinaciju tradicionalnih i naprednih, kvantitativnih pristupa. Razvijeni i analizirani modeli predstavljaju konkretan doprinos u tom smjeru, pružajući temelj za donošenje odluka koje nisu samo financijski isplative, već i informirane o rizicima koji ih prate.
