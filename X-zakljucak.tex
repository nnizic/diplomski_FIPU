\section{Diskusija i zaključak}
\label{chap:zakljucak}

Ovaj diplomski rad bavio se složenim problemom optimizacije portfelja projektnih aktivnosti u uvjetima nesigurnosti. S ciljem razvoja modela koji donositeljima odluka nudi ne samo profitabilna, već i robusna rješenja, razvijen je i evaluiran hibridni pristup koji integrira snagu genetskih algoritama za pretraživanje složenih prostora rješenja i Monte Carlo simulacije za kvantifikaciju rizika. Kroz sustavni, dvo-fazni eksperimentalni proces, provedena je prvo kalibracija parametara genetskog algoritma, a zatim i detaljna usporedna analiza triju optimizacijskih modela: osnovne metode nasumične pretrage, klasičnog genetskog algoritma usmjerenog isključivo na povrat na investiciju (ROI), te naprednog, više-kriterijskog hibridnog modela (GA+MC).

\subsection{Glavni nalazi i odgovori na istraživačka pitanja}
Provedeni eksperimenti pružili su jasne i empirijski utemeljene odgovore na istraživačka pitanja postavljena u uvodu rada.

\begin{enumerate}
    \item \textbf{Prvo, potvrđena je superiornost inteligentne pretrage.} Dok je nasumična pretraga bila donekle uspješna na jednostavnim problemima, njena učinkovitost drastično opada s porastom složenosti, čime je potvrđena hipoteza \texttt{H1}. Korištenje Monte Carlo simulacije kao samostalnog modela, bez inteligentnog mehanizma pretrage, potvrđuje ovaj zaključak: iako pouzdano kvantificira rizik za zadani portfelj, ne nudi smjernice za pronalazak boljih rješenja. S druge strane, genetski algoritmi su se pokazali značajno superiornijima u navigaciji kroz astronomski velik prostor rješenja, konzistentno pronalazeći portfelje s visokim \texttt{ROI}-em.
    \item \textbf{Drugo, dokazano je da hibridni model uspješno upravlja kompromisom između profita i rizika.} Hibridni \texttt{GA+MC} model, temeljen na \texttt{NSGA-II} algoritmu, uspješno je identificirao Paretov front optimalnih rješenja. Time je potvrđena hipoteza \texttt{H2}, jer model donositelju odluke ne nudi jedno, već čitav spektar strateških opcija koje balansiraju viši \texttt{ROI} s dužim procijenjenim trajanjem, i obrnuto. Vizualizacija Paretovog fronta pokazala se kao ključan alat za strateško odlučivanje, preoblikujući optimizacijski problem iz jednostavne potrage za jednim brojem u sofisticirani alat koji kvantificira ``cijenu'' smanjenja rizika.
    \item \textbf{Treće, utvrđeno je da stabilnost i robusnost modela ovise o kontekstu problema.} Analiza je otkrila ključan, nijansiran nalaz kojim je potvrđena hipoteza \texttt{H3}. U standardnim uvjetima, hibridni \texttt{GA+MC} model nudi rješenja čije je procijenjeno trajanje pouzdanije i stabilnije, s nižom standardnom devijacijom u odnosu na klasični \texttt{GA}. Međutim, pod ekstremnim pritiskom vrlo restriktivnog budžeta, njegova složenost postaje nedostatak. Više-kriterijski mehanizam pretrage, koji briljira u širokom prostoru rješenja, postaje neefikasan u iznimno suženom prostoru valjanih opcija, što dovodi do nestabilnosti i čestih neuspjeha. U tim uvjetima, jednostavniji, jedno-kriterijski \texttt{GA} pokazao se robusnijim, naglašavajući da ne postoji univerzalno ``najbolji'' algoritam.
\end{enumerate}

\subsection{Praktične implikacije i doprinos rada}
Ovaj rad ima značajne praktične implikacije za upravljanje projektima. Ključni doprinos jest demonstracija kako se apstraktni algoritamski koncepti mogu primijeniti za rješavanje stvarnih poslovnih problema.
\textbf{Prvo}, rad potvrđuje da je za efikasnu optimizaciju portfelja projekata primjena inteligentnih metaheuristika poput genetskih algoritama ne samo preporučljiva, već i nužna, s obzirom na kombinatornu eksploziju mogućih rješenja.
\textbf{Drugo}, prikazano je kako višeciljna optimizacija transformira proces donošenja odluka. Umjesto jednog ``crno-kutijskog'' odgovora, hibridni \texttt{GA+MC} model daje menadžerima potpunu sliku o kompromisima između \texttt{ROI}-a i rizika. To im omogućuje da donesu strateške odluke utemeljene na vlastitoj toleranciji na rizik. Primjerice, umjesto da prihvate najprofitabilniji, ali najduži projekt, mogu odabrati rješenje s 5\% manjim \texttt{ROI}-em, ali 20\% kraćim trajanjem, što je neprocjenjivo u dinamičnim okruženjima.
\textbf{Konačno}, analiza o robusnosti modela nudi ključnu lekciju: tehnološki najnapredniji alat nije uvijek najbolji u svim uvjetima. U situacijama s ekstremno ograničenim resursima, jednostavniji i fokusiraniji pristup može biti pouzdaniji i robusniji, što je vitalna informacija za svakog projektnog menadžera pri odabiru metodologije.

\subsection{Ograničenja rada i preporuke za budući rad}
U tijeku istraživanja uočena su i određena ograničenja koja definiraju domete ovog rada. Najznačajnije ograničenje je računalna zahtjevnost hibridnog \texttt{GA+MC} modela. Iako je \texttt{GA} učinkovit, tisuće Monte Carlo simulacija unutar svake iteracije značajno usporavaju proces, što ga čini nepraktičnim za rješavanje problema s izuzetno velikim brojem projekata. Drugi izazov je korištenje sintetičkih podataka. Iako su parametri odabrani da budu realistični, validacija modela na stvarnim, povijesnim podacima iz industrije predstavljala bi neprocjenjiv sljedeći korak. Konačno, krhkost naprednog modela pod ekstremnim ograničenjima upućuje na to da su potrebna daljnja istraživanja s ciljem pronalaska mehanizama za poboljšanje njegove robusnosti u takvim uvjetima.
Na temelju provedenog istraživanja i uočenih ograničenja, izdvaja se nekoliko pravaca za budući rad.

\begin{itemize}
    \item \textbf{Hibridizacija s drugim metaheuristikama:} Jedan od smjerova je kombiniranje s algoritmima rojeva čestica (\texttt{PSO}) ili simuliranog kaljenja, radi potencijalnog poboljšanja brzine konvergencije. Ti bi se algoritmi mogli pokazati bržima u istraživanju prostora rješenja, dok bi \texttt{MC} simulacija ostala ključna za procjenu rizika.
    \item \textbf{Primjena strojnog učenja:} Drugi zanimljiv pravac je korištenje strojnog učenja, primjerice regresijskih modela, za preciznije predviđanje distribucija nesigurnosti iz povijesnih podataka. To bi smanjilo oslanjanje na subjektivne procjene projektnih menadžera i povećalo objektivnost analize.
    \item \textbf{Razvoj interaktivnog softverskog alata:} Za maksimalnu praktičnu primjenjivost, preporučuje se razvoj softverskog alata s intuitivnim grafičkim korisničkim sučeljem. Takav alat ne samo da bi automatizirao proces optimizacije, već bi omogućio i interaktivnu vizualizaciju Paretovog fronta, dajući menadžerima moćan alat za istraživanje kompromisa u realnom vremenu.
\end{itemize}

Zaključno, ovaj rad je potvrdio da primjena naprednih algoritamskih rješenja nudi značajan potencijal za unapređenje procesa upravljanja projektima u uvjetima nesigurnosti. Učinkovito upravljanje, kako ističe Kerzner \cite{Kerzner2017}, u suvremenom okruženju zahtijeva upravo kombinaciju tradicionalnih i naprednih, kvantitativnih pristupa. Razvijeni i analizirani modeli predstavljaju konkretan doprinos u tom smjeru, pružajući temelj za donošenje odluka koje nisu samo financijski isplative, već i informirane o rizicima koji ih prate.
