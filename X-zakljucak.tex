\section{Zaključak}

U ovom diplomskom radu predstavili smo kompleksan pristup optimizaciji raspodjele projektnih aktivnosti koristeći kombinaciju genetskih algoritama i Monte Carlo simulacije. Cilj je bio razviti model koji uzima u obzir nesigurnost u trajanju, troškovima i vrijednosti zadataka, te u okviru zadanih ograničenja vremena, budžeta i raspoloživih resursa maksimizira ukupnu vrijednost projekta.

Kroz detaljnu analizu problematike i pregled postojeće literature, identificirali smo ključne izazove u upravljanju projektima, posebice u segmentu neizvjesnosti i složenosti optimizacije. Implementacijom metaheurističkih metoda, u ovom slučaju genetskih algoritama \cite{Goldberg1989, Mitchell1998}, omogućili smo efikasno pretraživanje velikog prostora rješenja, dok je Monte Carlo simulacija služila za kvantitativno modeliranje rizika i nesigurnosti \cite{Miller2009, Avlijas2008}, dajući time realističniju procjenu performansi optimizacijskog rješenja.

Praktična implementacija rezultirala je modelom koji omogućuje donošenje informiranih odluka u planiranju i upravljanju projektima, pružajući projektnim menadžerima alate za bolje usklađivanje ciljeva i ograničenja. Pokazalo se da je kombinacija ovih metoda učinkovita u pronalasku balansiranih rješenja koja maksimiziraju povrat ulaganja, uz minimizaciju rizika od prekoračenja budžeta ili rokova.

Iako su postignuti rezultati zadovoljavajući, postoje brojna područja za buduća istraživanja i unaprjeđenja, među kojima izdvajamo:
\begin{itemize}
    \item Proširenje modela na dinamičke uvjete projekata koji se mijenjaju tijekom vremena, uključujući nepredvidive vanjske utjecaje.
    \item Integracija dodatnih metaheurističkih i hibridnih algoritama, poput algoritama rojčaste inteligencije ili simuliranog kaljenja, radi poboljšanja kvalitete rješenja \cite{Gandomi2013}.
    \item Primjena tehnika strojnog učenja za preciznije predviđanje distribucija nesigurnosti i automatsku adaptaciju parametara optimizacije.
    \item Razvoj softverskih alata s intuitivnim korisničkim sučeljem za praktičnu primjenu predloženih metoda u realnim projektnim okruženjima.
\end{itemize}

Zaključno, ovaj rad potvrđuje važnost primjene naprednih algoritamskih rješenja u upravljanju projektima, posebno u uvjetima nesigurnosti, te doprinosi boljem razumijevanju i praktičnoj primjeni optimizacijskih i simulacijskih metoda u području projektne ekonomike i menadžmenta. Kao što ističe Kerzner \cite{Kerzner2017}, učinkovito upravljanje projektima u suvremenom okruženju zahtijeva kombinaciju tradicionalnih i naprednih pristupa, a naš model pruža značajan doprinos u tom smjeru.


