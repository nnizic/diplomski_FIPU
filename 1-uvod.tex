\section{Uvod}

\subsection{Uvodni dio}
Jedan od modela koji dobro opisuje izazove alokacije resursa je \textit{problem ruksaka} (\textit{Knapsack Problem}), gdje se ograničeni resursi moraju raspodijeliti na način koji maksimizira ukupnu korist \cite{Kellerer2004}. U kontekstu projektnih aktivnosti, to znači odabrati skup zadataka koji će donijeti najveći povrat ulaganja (ROI) uz poštivanje vremenskih i resursnih ograničenja. Tradicionalne metode poput PERT-a (\textit{Program Evaluation and Review Technique}) i CPM-a (\textit{Critical Path Method}) \cite{Malcolm1959,Kerzner2017} pružaju okvir za planiranje i praćenje projekata, no često zanemaruju složene kombinacije zadataka i inherentnu neizvjesnost u procjenama trajanja i troškova.

Pregled literature pokazuje da se genski algoritmi (GA) uspješno primjenjuju na NP-teške optimizacijske probleme \cite{Holland1975,Goldberg1989}, dok Monte Carlo simulacije (MC) omogućuju modeliranje i kvantifikaciju neizvjesnosti \cite{Metropolis1949,Rubinstein2016}. Međutim, u literaturi postoji ograničen broj radova koji integriraju ove dvije metode u jedinstven hibridni pristup optimizaciji projektnih aktivnosti \cite{Deb2002,Zhang2011}. Praznina se očituje u nedostatku sustavne analize kako kombinacija GA i MC može povećati robusnost rješenja u uvjetima visoke neizvjesnosti i složenih resursnih ograničenja.

\subsection{Motivacija}
Motivacija za ovaj rad proizlazi iz činjenice da se u stvarnim projektnim uvjetima planovi rijetko odvijaju točno onako kako su prvotno predviđeni \cite{Kerzner2017}. Promjene u dostupnosti resursa, nepredviđeni rizici, promjene zahtjeva te pogrešne procjene trajanja zadataka česti su razlozi odstupanja od plana. Klasične metode planiranja projekata, iako korisne, često ne uzimaju u obzir dinamičke promjene i stohastičku prirodu projektnih parametara \cite{PMI2021}.

Upravo zbog toga potrebne su metode koje ne samo da optimiziraju raspodjelu resursa, već i uvažavaju nesigurnosti te omogućuju procjenu rizika povezanih s odabranim rješenjima. Genski algoritmi pružaju snažan alat za globalnu optimizaciju složenih problema pretrage \cite{Goldberg1989}, dok Monte Carlo simulacije omogućuju procjenu varijabilnosti i rizika rješenja \cite{Rubinstein2016}. Kombiniranjem ovih metoda moguće je razviti pristup koji generira kvalitetna rješenja, a pritom osigurava njihovu robusnost u uvjetima nesigurnosti \cite{Zhang2011}.

\subsection{Rizici i nesigurnosti u projektnom upravljanju}
Rizici u projektnom upravljanju obuhvaćaju sve događaje ili uvjete koji, ako se pojave, mogu negativno utjecati na ciljeve projekta \cite{Hillson2009}. Oni mogu biti tehničke, organizacijske, financijske ili tržišne prirode, a često su međusobno povezani \cite{PMI2021}. Nesigurnosti proizlaze iz nepotpunih informacija, promjenjivih uvjeta i nemogućnosti preciznog predviđanja budućih događaja \cite{Smith2014}. Upravljanje rizicima uključuje njihovu identifikaciju, procjenu i razvoj strategija odgovora \cite{Hillson2009}.

Monte Carlo simulacije predstavljaju jedan od najučinkovitijih pristupa kvantitativnoj procjeni rizika \cite{Vose2008}. Kroz generiranje velikog broja scenarija temeljenih na slučajnim varijacijama ulaznih parametara, moguće je procijeniti distribuciju mogućih ishoda i vjerojatnosti ostvarenja ciljeva projekta \cite{Rubinstein2016}.

\subsection{Ciljevi i istraživačka pitanja}
Ovaj diplomski rad ima za cilj razviti i evaluirati hibridni GA-MC model za optimizaciju raspodjele projektnih aktivnosti. Doprinos rada je trojak:
\begin{itemize}
    \item \textbf{Teorijski doprinos}: povezivanje optimizacijskih tehnika s metodama procjene neizvjesnosti u kontekstu projektnih aktivnosti.
    \item \textbf{Metodološki doprinos}: razvoj integriranog modela koji povezuje evolucijsku optimizaciju i statističku simulaciju.
    \item \textbf{Praktični doprinos}: demonstracija primjenjivosti pristupa na sintetičkim projektnim podacima i analiza performansi modela.
\end{itemize}

Istraživačka pitanja na koja ovaj rad želi odgovoriti su:
\begin{enumerate}
    \item Može li hibridni GA-MC pristup postići bolje rezultate optimizacije u odnosu na samostalnu primjenu GA ili MC?
    \item U kojoj mjeri Monte Carlo simulacija može poboljšati robusnost rješenja dobivenih genskim algoritmom?
    \item Kakav je utjecaj kombiniranog pristupa na stabilnost i pouzdanost optimizacije u uvjetima neizvjesnosti?
\end{enumerate}

Ciljevi istraživanja su:
\begin{itemize}
    \item Dizajnirati i implementirati integrirani GA-MC model.
    \item Evaluirati performanse modela na različitim scenarijima.
    \item Usporediti rezultate hibridnog pristupa s rezultatima pojedinačnih metoda.
\end{itemize}

Sinergija između genskih algoritama i Monte Carlo simulacija proizlazi iz njihove komplementarne prirode: GA učinkovito pretražuje veliki prostor mogućih rješenja i pronalazi visokokvalitetne kandidate \cite{Goldberg1989}, dok MC kvantificira rizik i procjenjuje varijabilnost tih rješenja kroz stohastičko modeliranje \cite{Rubinstein2016}. Kombinacija ovih metoda rezultira robusnim pristupom koji ne samo da optimizira projektne aktivnosti, već i uzima u obzir realne nesigurnosti u planiranju.

Struktura rada organizirana je na sljedeći način:
\begin{itemize}
    \item \textbf{Poglavlje 2} daje pregled relevantne literature i teorijskih osnova problema.
    \item \textbf{Poglavlje 3} opisuje korištenu metodologiju, uključujući principe rada GA i MC.
    \item \textbf{Poglavlje 4} prikazuje implementaciju hibridnog modela.
    \item \textbf{Poglavlje 5} opisuje plan i rezultate eksperimentalnog vrednovanja modela.
    \item \textbf{Poglavlje 6} sadrži zaključak i preporuke za budući rad.
\end{itemize}

