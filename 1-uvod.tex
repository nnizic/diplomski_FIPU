\section{Uvod}
\label{chap:uvod}


U suvremenom projektnom upravljanju, alokacija ograničenih resursa na niz konkurentskih aktivnosti predstavlja jedan od fundamentalnih izazova. Ovaj problem optimizacije može se elegantno modelirati kroz klasični \textit{problem ruksaka} (\textit{Knapsack Problem}), gdje je cilj odabrati podskup stavki na način koji maksimizira ukupnu vrijednost, a da se pritom ne prekorači zadani kapacitet \cite{Kellerer2004}. U domeni projekata, to se preslikava na odabir portfelja aktivnosti koji će generirati najveći mogući povrat na investiciju (ROI), uz strogo poštivanje proračunskih, vremenskih i drugih resursnih ograničenja.

Iako tradicionalne metode poput PERT-a (\textit{Program Evaluation and Review Technique}) i CPM-a (\textit{Critical Path Method}) \cite{Malcolm1959,Kerzner2017} pružaju nezamjenjiv okvir za planiranje i praćenje, njihova primjena nailazi na poteškoće u dinamičnim okruženjima. One su često determinističke prirode i teško se nose s dva ključna izazova: kombinatornom eksplozijom mogućih rješenja i inherentnom nesigurnošću koja prati procjene trajanja i troškova.

Pregled literature ukazuje na to da su genetski algoritmi (GA) iznimno učinkoviti u rješavanju NP-teških optimizacijskih problema \cite{Holland1975,Goldberg1989}, dok Monte Carlo (MC) simulacije predstavljaju zlatni standard za modeliranje i kvantifikaciju rizika \cite{Metropolis1949,Rubinstein2016}. Unatoč tome, primjetan je nedostatak radova koji sustavno istražuju sinergiju ovih dviju moćnih tehnika. Upravo se u tom prostoru pozicionira ovaj rad, s ciljem popunjavanja uočene praznine kroz razvoj i rigoroznu evaluaciju hibridnog GA-MC modela. Fokus nije samo na kombinaciji, već na stvaranju sustavnog eksperimentalnog okvira za kalibraciju i usporednu analizu takvih modela, kako bi se precizno kvantificirao utjecaj ove sinergije na robusnost i kvalitetu konačnih rješenja.

\subsection{Motivacija i značaj istraživanja}

Motivacija za ovo istraživanje proizlazi iz fundamentalne razlike između projektnog plana i stvarne izvedbe. U praksi, projekti se rijetko odvijaju točno prema početnim predviđanjima \cite{Kerzner2017}. Fluktuacije u dostupnosti resursa, nepredviđeni događaji, izmjene u zahtjevima dionika te inherentna nepreciznost početnih procjena česti su uzroci odstupanja, što klasične metode planiranja čini nedovoljno fleksibilnima \cite{PMI2021}.

Stoga se javlja potreba za pristupima koji nadilaze puku alokaciju resursa i ulaze u domenu strateškog upravljanja nesigurnošću. Potrebne su metode koje ne samo da pronalaze matematički optimalan raspored, već vrednuju rješenja i prema njihovoj otpornosti na nepredviđene okolnosti. Ovdje komplementarna priroda genetskih algoritama i Monte Carlo simulacija dolazi do punog izražaja. Genski algoritmi pružaju mehanizam za efikasno pretraživanje astronomski velikog prostora mogućih portfelja \cite{Goldberg1989}, dok Monte Carlo simulacije služe kao "simulator stvarnosti", testirajući svako potencijalno rješenje u tisućama mogućih budućnosti kako bi se procijenila njegova stvarna varijabilnost i rizik \cite{Rubinstein2016}.

Kombiniranjem ovih metoda, ovaj rad prelazi s tradicionalnog pitanja "Koji je najbolji plan?" na puno značajnije pitanje: "Koji je najrobusniji plan koji nudi najbolji kompromis između profita i rizika?".

\subsection{Izazov nesigurnosti u projektnom upravljanju}

Rizici u projektnom upravljanju obuhvaćaju sve događaje ili uvjete koji, ako se pojave, mogu negativno utjecati na ciljeve projekta \cite{Hillson2009}. Oni mogu biti tehničke, organizacijske, financijske ili tržišne prirode, a često su međusobno povezani \cite{PMI2021}. Nesigurnosti, s druge strane, proizlaze iz nepotpunih informacija, promjenjivih uvjeta i fundamentalne nemogućnosti preciznog predviđanja budućih događaja \cite{Smith2014}. Učinkovito upravljanje projektima stoga zahtijeva sustavan pristup identifikaciji, procjeni i razvoju strategija odgovora na rizike i nesigurnosti \cite{Hillson2009}.

U domeni kvantitativne analize, Monte Carlo simulacije predstavljaju jedan od najmoćnijih pristupa \cite{Vose2008}. Metoda se temelji na generiranju velikog broja mogućih scenarija temeljenih na specificiranim distribucijama vjerojatnosti za neizvjesne ulazne parametre. Analizom distribucije dobivenih ishoda, moguće je s visokom pouzdanošću procijeniti, primjerice, vjerojatnost završetka projekta unutar zadanog roka ili budžeta \cite{Rubinstein2016}.

\subsection{Ciljevi, doprinos i struktura rada}

Primarni cilj ovog diplomskog rada jest razviti i kroz sustavne eksperimente evaluirati hibridni GA-MC model za optimizaciju portfelja projektnih aktivnosti. Svrha modela je pružiti podršku u donošenju odluka koje su informirane ne samo o potencijalnoj dobiti, već i o pripadajućem riziku. Doprinos rada ostvaren je na teorijskoj, metodološkoj i praktičnoj razini. Na teorijskoj razini, rad povezuje koncepte iz evolucijskog računarstva s kvantitativnim metodama upravljanja rizikom u jedinstven okvir primijenjen na projektno upravljanje. S metodološkog stajališta, razvijen je i validiran cjelovit dvo-fazni proces za evaluaciju složenih hibridnih modela, koji uključuje fazu kalibracije i fazu usporedne analize. Konačno, praktični doprinos očituje se u demonstraciji primjenjivosti pristupa na sintetičkim podacima, čime se dobiva uvid u performanse i ograničenja različitih optimizacijskih strategija.

Kako bi se ostvario primarni cilj, istraživanje je vođeno sljedećim ključnim pitanjima, od kojih svako predstavlja zaseban sloj analize:

\begin{enumerate}
\item U kojim uvjetima i prema kojim metrikama (npr. profitabilnost, rizik trajanja, stabilnost) hibridni GA-MC pristup postiže superiorne rezultate u odnosu na samostalnu primjenu GA ili MC modela? \\
Ovo pitanje predstavlja temeljnu usporednu analizu. Njegova svrha nije samo potvrditi superiornost hibridnog modela, već i detaljno kvantificirati tu superiornost kroz višestruke metrike – od maksimalnog postignutog ROI-a, preko procijenjenog trajanja, pa sve do pouzdanosti samih rezultata. Odgovor na ovo pitanje omogućit će nam da jasno razumijemo u kojim situacijama je kombinirani pristup neprikosnoven, a kada bi mogao biti prekomjeran.

\item U kojoj mjeri integracija Monte Carlo simulacije u proces evaluacije može poboljšati robusnost rješenja dobivenih genetskim algoritmom? \\
Ovo pitanje zadire u samu srž sinergije dviju tehnika. Njegov cilj je testirati temeljnu hipotezu da algoritam koji uči iz stohastičkih, a ne samo determinističkih podataka, stvara rješenja koja su otpornija na inherentne nesigurnosti projektnih varijabli. Kroz eksperimente, nastojat ćemo dokazati da se rješenja hibridnog modela, iako možda ne dosežu apsolutno najveći ROI, ponašaju predvidljivije i manje su podložna negativnim iznenađenjima u stvarnim uvjetima.

\item Kakav je utjecaj kombiniranog pristupa na stabilnost i pouzdanost optimizacije u uvjetima neizvjesnosti? \\
Treće pitanje fokusira se na praktičnu vrijednost modela za donositelja odluka. Stabilnost i pouzdanost, mjerene standardnom devijacijom, kritični su parametri koji određuju hoće li se algoritam koristiti kao pouzdan alat ili kao nepouzdana "crna kutija". Odgovor na ovo pitanje otkrit će koliko je hibridni model konzistentan u pronalaženju kvalitetnih rješenja kroz više nezavisnih pokretanja, čime će se potvrditi njegova praktična primjenjivost za strateško planiranje.
\end{enumerate}

Sinergija između genskih algoritama i Monte Carlo simulacija proizlazi iz njihove komplementarne prirode: GA učinkovito pretražuje veliki prostor mogućih rješenja i pronalazi visokokvalitetne kandidate \cite{Goldberg1989}, dok MC kvantificira rizik i procjenjuje varijabilnost tih rješenja kroz stohastičko modeliranje \cite{Rubinstein2016}.

\subsection{Od pristupa do rješenja: put do optimalnih kompromisa}

Metodološki pristup ovog rada nije usmjeren samo na demonstraciju superiornosti jedne tehnike, već na dubinsko razumijevanje njezinih mehanizama i ograničenja. Eksperimentalni dio započinje ablacijskom studijom, gdje će se detaljno analizirati utjecaj ključnih genetskih operatora poput križanja i mutacije. Ta studija će empirijski pokazati da, suprotno intuiciji, strategija širine pretrage (veća populacija) daje značajno bolje i stabilnije rezultate od strategije dubine pretrage (više generacija). Ovaj nalaz poslužit će kao temelj za kalibraciju "šampionske" konfiguracije genetskog algoritma za kasniju usporedbu.

Nakon kalibracije, rad će usporediti tri modela: nasumičnu pretragu, klasični \texttt{GA} fokusiran na \texttt{ROI}, i napredni hibridni \texttt{GA+MC} model. Usporedba će otkriti kako se svaki model ponaša u tri scenarija složenosti i budžetskih ograničenja. Očekuje se da će klasični \texttt{GA} dokazati svoju moć kao iznimno učinkovit "profitni maksimizator", dok će nasumična pretraga poslužiti kao jasan dokaz da je za rješavanje problema realne veličine primjena metaheuristika nužna, a ne samo poželjna.

Srž ovog istraživanja leži u evaluaciji hibridnog \texttt{GA+MC} modela. On je razvijen s ciljem da nadilazi tradicionalni "crna kutija" pristup, koji isporučuje samo jedno, optimalno rješenje. Umjesto toga, hibridni model će generirati textbf{Paretov front} – skup ne-dominiranih rješenja koja vizualno predstavljaju kompromis (\textit{trade-off}) između profitabilnosti (\texttt{ROI}) i rizika (trajanje projekta). Rad će dokazati da je ova vizualizacija ključni alat za strateško odlučivanje, omogućujući donositelju odluka da odabere rješenje koje najbolje odgovara njegovoj toleranciji na rizik.

Konačno, analiza će razotkriti i neočekivan, ali ključan nalaz: složenost ne mora uvijek značiti superiornost. Pokazat će se da u ekstremno restriktivnim scenarijima, gdje su resursi iznimno ograničeni, napredni hibridni model postaje ranjiv i nestabilan, dok jednostavniji, jedno-kriterijski \texttt{GA} pokazuje iznimnu robusnost. Ovaj nalaz naglašava da odabir modela mora biti kontekstualan i ovisiti o specifičnim uvjetima i ograničenjima. S ovim proširenjem, uvod će ne samo predstaviti problem, već i stvoriti temelj za duboko razumijevanje svih ključnih rezultata koji slijede.

Ostatak rada organiziran je kako slijedi. \textbf{Drugo poglavlje} pruža pregled relevantne literature i teorijskih osnova, uključujući problem ruksaka, genetske algoritme i metode modeliranja nesigurnosti. \textbf{Treće poglavlje} detaljno opisuje matematički i konceptualni model problema koji se rješava. \textbf{Četvrto poglavlje} ulazi u detalje implementacije, opisujući arhitekturu razvijenog softverskog sustava i korištene programske biblioteke. \textbf{Peto poglavlje} čini srž rada, prikazujući dvo-fazni eksperimentalni dizajn, analizu dobivenih rezultata i detaljnu diskusiju o performansama testiranih modela. Konačno, \textbf{šesto poglavlje} donosi sintezu cjelokupnog rada, sažima ključne doprinose, osvrće se na ograničenja provedenog istraživanja i nudi preporuke za budući rad.

