\section{Uvod}

Upravljanje projektima je ključna aktivnost u brojnim industrijama, od građevine i IT-a do farmaceutike i financija. Jedan od najzahtjevnijih aspekata upravljanja projektima jest učinkovita raspodjela aktivnosti i resursa kroz vrijeme, pri čemu se mora zadovoljiti niz ograničenja, uključujući budžet, vremenski rok, kapacitete resursa i međusobne ovisnosti između zadataka. U složenim projektima s velikim brojem aktivnosti, tradicionalni pristupi često nisu dostatni jer ne uspijevaju adresirati neizvjesnosti i varijabilnost koje prate realne projekte.

\subsection{Motivacija}

Raspodjela projektnih aktivnosti i resursa u uvjetima nesigurnosti i ograničenja predstavlja NP-težak problem, što znači da se broj mogućih kombinacija rješenja eksponencijalno povećava s veličinom problema. Tradicionalne metode kao što su CPM (Critical Path Method) i PERT (Program Evaluation and Review Technique) podrazumijevaju determinističke vremenske procjene i ne uključuju varijabilnost stvarnih uvjeta, što može dovesti do suboptimalnih ili čak neizvedivih planova.

Potreba za metodama koje mogu obuhvatiti stohastičku prirodu trajanja aktivnosti, dinamiku projektnog okruženja i složene međusobne odnose među aktivnostima, motivira primjenu naprednih optimizacijskih i simulacijskih tehnika.

\subsection{Rizici i nesigurnosti u projektnom upravljanju}

U stvarnim projektima, trajanja aktivnosti često nisu poznata unaprijed s potpunom sigurnošću. Kašnjenja, nedostatak resursa, promjene u specifikacijama ili nepredviđene okolnosti mogu značajno utjecati na tijek projekta. Zbog toga je važno uključiti kvantitativne metode za procjenu rizika i analizu nesigurnosti. Upravo tu se Monte Carlo simulacija ističe kao snažan alat koji omogućuje evaluaciju raspodjele mogućih ishoda i procjenu vjerojatnosti završetka projekta unutar zadanih rokova.

\subsection{Monte Carlo simulacija i genetski algoritmi}

Monte Carlo simulacija koristi slučajne uzorke za kvantificiranje nesigurnosti u modelima i omogućuje realističnije procjene vremenskih i troškovnih distribucija. U kontekstu projektnog upravljanja, ova metoda može simulirati tisuće mogućih scenarija izvedbe aktivnosti na temelju probabilističkih ulaza (npr. optimističnog, realnog i pesimističnog trajanja).

Genetski algoritmi (GA) predstavljaju jednu od najčešće korištenih metaheurističkih metoda za rješavanje složenih problema optimizacije. Temelje se na principima evolucije i prirodne selekcije te su učinkoviti u pretraživanju velikih prostora rješenja, što ih čini pogodnima za optimizaciju projektnih rasporeda.

\subsection{Cilj rada}

Cilj ovog rada je razviti model koji kombinira genetski algoritam s Monte Carlo simulacijom radi dobivanja robusnog plana raspodjele aktivnosti u projektu. Kombinacija ovih dviju metoda omogućava simultano:
\begin{itemize}
    \item optimiziranje projektnih rasporeda u uvjetima složenih ograničenja,
    \item kvantifikaciju rizika i nesigurnosti u izvedbi projekta,
    \item donošenje boljih odluka u upravljanju resursima.
\end{itemize}

Predloženi pristup testira se na simuliranim podacima i evaluira s obzirom na pouzdanost završetka projekta unutar vremenskog roka i efikasnost raspodjele resursa.

