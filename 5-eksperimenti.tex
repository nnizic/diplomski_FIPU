\section{Eksperimenti}

Cilj ovog poglavlja je provesti eksperimentalnu evaluaciju razvijenog modela optimizacije raspodjele projektnih aktivnosti koji kombinira genetske algoritme (GA) i Monte Carlo simulaciju (MC). U ovoj fazi rada eksperimenti će biti okvirno opisani, dok će se stvarni rezultati i detaljne analize dodati nakon provedbe testiranja.

\subsection{Testni podaci}
Za potrebe eksperimenta koristit će se sintetički skupovi podataka koji će uključivati:
\begin{itemize}
    \item različite vrijednosti projektnih aktivnosti,
    \item različite troškove resursa,
    \item varijabilna trajanja aktivnosti (optimistično, realno, pesimistično).
\end{itemize}
Takav pristup omogućava kontrolu nad parametrima te usporedivost rezultata između različitih scenarija.

\subsection{Scenariji testiranja}
Eksperimenti će biti podijeljeni u tri osnovna scenarija:
\begin{enumerate}
    \item \textbf{Samo GA} — Genetski algoritam optimizira raspodjelu bez uključivanja Monte Carlo simulacije (baseline pristup).
    \item \textbf{Samo MC} — Monte Carlo simulacija koristi se za procjenu trajanja i uspješnosti bez GA optimizacije.
    \item \textbf{Kombinacija GA + MC} — Predloženi hibridni pristup gdje se GA koristi za optimizaciju, a MC za evaluaciju potencijalnih rješenja (finalni model).
\end{enumerate}

\subsection{Metodologija}
Za svaki scenarij provodit će se više ponavljanja kako bi se smanjio utjecaj slučajnih varijacija.  
Planirane metrike koje će se prikupljati:
\begin{itemize}
    \item \textbf{Broj uspješnih projekata} — koliko projekata je završilo unutar planiranog roka i budžeta.
    \item \textbf{Ukupna ROI vrijednost} — ukupni povrat investicije iz optimizirane raspodjele.
    \item \textbf{Stabilnost rješenja} — varijacija rezultata kroz ponavljanja eksperimenta.
\end{itemize}

\subsection{Plan prezentacije rezultata}
Nakon provedbe eksperimenata, rezultati će biti prikazani:
\begin{itemize}
    \item tablicama (kvantitativni rezultati i usporedbe),
    \item grafovima (vizualizacija trendova i distribucija),
    \item opisnim analizama (interpretacija dobivenih rezultata).
\end{itemize}

Sljedeći grafikon prikazat će usporedbu prosječne uspješnosti između triju scenarija:

\begin{figure}[H]
    \centering
    \includegraphics[width=0.85\textwidth]{slike/usporedba_scenarija.png}
    \caption{Primjer usporedbe prosječne uspješnosti za tri scenarija (primjer, podaci privremeni).}
    \label{fig:usporedba_scenarija}
\end{figure}
