\section{Teorijska podloga}
Ovo poglavlje pruža pregled temeljnih teorijskih koncepata ključnih za razumijevanje predloženog modela optimizacije projektnih aktivnosti. Detaljno će se objasniti Problem ruksaka kao osnova za formulaciju problema raspodjele resursa, Genetski algoritmi kao optimizacijska metaheuristika, te Monte Carlo simulacija i metode procjene trajanja kao alati za modeliranje i analizu nesigurnosti u projektnom upravljanju.

\subsection{Knapsack problem}
Problem ruksaka (engl. Knapsack Problem) jedan je od najpoznatijih i najčešće proučavanih problema kombinatorne optimizacije, svrstan u klasu NP-teških problema \cite{Goldberg1989}. U osnovnoj verziji, cilj je odabrati podskup objekata s pridruženom težinom i vrijednošću, s ciljem maksimiziranja ukupne vrijednosti odabranih objekata, pri čemu njihova ukupna težina ne smije prelaziti zadani kapacitet ruksaka. Formalno, za skup od n objekata, gdje svaki objekt i ima težinu $w_i$ i vrijednost $v_i$, te uz zadani kapacitet ruksaka W, cilj je maksimizirati funkciju:
$$
\max \sum_{i=1}^n v_i x_i \quad \text{uz ograničenje} \quad \sum_{i=1}^n w_i x_i \leq W, x_i \in \{0,1\}
$$
gdje je $x_i=1$ ako je objekt i odabran, a $x_i=0$ ako nije.

U kontekstu upravljanja projektima, ovaj se problem često pojavljuje u složenijim varijantama, poput višedimenzionalnog problema ruksaka (Multi-Dimensional Knapsack Problem – MDKP). U MDKP-u, projektne aktivnosti se mogu interpretirati kao objekti s određenom vrijednošću (npr. strateška važnost, povrat investicije – ROI), dok svaka aktivnost troši više vrsta resursa (npr. vrijeme, budžet, broj radnika), koji predstavljaju različite dimenzije "težine". Kapacitet ruksaka tada predstavlja ukupnu raspoloživost svakog od resursa. MDKP se stoga koristi kao snažan model za optimalnu raspodjelu ograničenih, višestrukih resursa među konkurentskim projektnim aktivnostima.

\subsection{Genetski algoritmi}
Genetski algoritmi (GA) su moćne metaheurističke optimizacijske metode inspirirane procesima prirodne selekcije i evolucije \cite{Goldberg1989, Mitchell1998}. Pripadaju široj klasi evolucijskih algoritama i iznimno su učinkoviti u rješavanju složenih optimizacijskih problema s velikim i nepreglednim prostorom rješenja, posebno onih NP-teških, za koje klasične metode nisu praktične \cite{Gandomi2013, Kaveh2012}.

Osnovni princip GA leži u simulaciji evolucije populacije potencijalnih rješenja. Svako rješenje problema kodira se kao kromosom (u ovom radu, binarni niz), a populacija kromosoma se iterativno poboljšava kroz generacije primjenom genetskih operatora. Tijek genetskog algoritma uključuje:
\begin{enumerate}
    \item \textbf{Inicijalizacija populacije:} Generira se početni skup nasumičnih kromosoma.
    \item \textbf{Evaluacija funkcije cilja (fitness):} Svakom kromosomu dodjeljuje se vrijednost pogodnosti (fitnessa) koja odražava kvalitetu rješenja. U ovisnosti o cilju, funkcija pogodnosti može biti jedno-kriterijska (npr. maksimizacija ROI-a) ili više-kriterijska. U naprednijim modelima, kao što je hibridni model razvijen u ovom radu, fitness funkcija može uključivati i rezultate Monte Carlo simulacije kako bi se procijenila robusnost rješenja.
    \item \textbf{Selekcija roditelja:} Kromosomi s višim fitnessom imaju veću vjerojatnost da budu odabrani kao roditelji za stvaranje sljedeće generacije.
    \item \textbf{Križanje (crossover):} Dva odabrana roditelja kombiniraju se kako bi se stvorili novi potomci, prenoseći genetski materijal i istražujući nove dijelove prostora rješenja.
    \item \textbf{Mutacija:} Slučajne, male promjene unose se u kromosome potomaka kako bi se održala genetska raznolikost populacije i izbjegla prerana konvergencija.
    \item \textbf{Zamjena populacije:} Nova generacija potomaka zamjenjuje dio ili cijelu staru populaciju, i proces se ponavlja dok se ne ispuni kriterij zaustavljanja (npr. zadan broj generacija).
\end{enumerate}

\subsection{Monte Carlo simulacija}
Monte Carlo simulacija (MCS) je računska metoda koja koristi nasumično uzorkovanje za procjenu ponašanja složenog sustava ili procesa, posebno kada je analitičko rješenje teško ili nemoguće. Njena je primarna prednost sposobnost modeliranja nesigurnosti i rizika u sustavima s probabilističkim ulaznim varijablama \cite{Vose2008}. U kontekstu projektnog upravljanja, MCS je vrijedan alat za procjenu vjerojatnih ishoda projekta, poput trajanja i troškova, uzimajući u obzir varijabilnost aktivnosti \cite{Miller2009, Avlijas2008}.

Ključni elementi MCS uključuju:
\begin{itemize}
    \item \textbf{Definiranje slučajnih varijabli:} Identificiraju se ulazne varijable čija je vrijednost neizvjesna (npr. trajanje aktivnosti).
    \item \textbf{Odabir distribucije vjerojatnosti:} Za svaku varijablu odabire se distribucija koja najbolje opisuje njeno ponašanje.
    \item \textbf{Generiranje nasumičnih uzoraka:} Velik broj uzoraka generira se iz odabranih distribucija.
    \item \textbf{Provođenje simulacije:} Za svaki skup uzoraka provodi se izračun modela (npr. zbrajanje trajanja aktivnosti).
    \item \textbf{Analiza rezultata:} Nakon velikog broja iteracija, prikupljeni podaci se analiziraju statistički kako bi se dobila distribucija mogućih ishoda.
\end{itemize}

\subsection{Modeliranje nesigurnosti trajanja: PERT i Trokutasta distribucija}
Metodologija PERT (Program Evaluation and Review Technique) uvela je praksu korištenja tri vremenske procjene za aktivnosti s neizvjesnim trajanjem \cite{Malcolm1959}:
\begin{itemize}
    \item \textbf{$T_o$} – optimistična procjena trajanja (najkraće moguće trajanje).
    \item \textbf{$T_m$} – najvjerojatnija procjena trajanja (očekivano trajanje).
    \item \textbf{$T_p$} – pesimistična procjena trajanja (najduže moguće trajanje).
\end{itemize}
Dok tradicionalna PERT metoda koristi ove tri točke za izračun parametara Beta distribucije, u modernoj praksi upravljanja rizikom, a posebno u Monte Carlo simulacijama, često se koristi Trokutasta distribucija zbog svoje jednostavnosti i intuitivnosti \cite{Law2015}.

Trokutasta distribucija je kontinuirana distribucija vjerojatnosti definirana s tri parametra: minimum ($a$), maksimum ($b$) i najvjerojatnija vrijednost ($c$), što direktno odgovara procjenama $T_o$, $T_p$ i $T_m$. Njena je glavna prednost što ne zahtijeva opsežne povijesne podatke, već se može temeljiti na stručnom iskustvu, što je čini iznimno pogodnom za projektno planiranje. Slučajne vrijednosti generirane iz ove distribucije nalaze se unutar intervala [$T_o$, $T_p$], s najvećom vjerojatnošću pojavljivanja oko vrijednosti $T_m$. Prosječna vrijednost (očekivano trajanje) za Trokutastu distribuciju računa se jednostavnom formulom:
$$
E(T) = \frac{T_o + T_m + T_p}{3}
$$
Upravo je Trokutasta distribucija, zbog navedenih prednosti, odabrana kao temelj za modeliranje nesigurnosti trajanja aktivnosti u Monte Carlo simulacijama provedenim u ovom radu.
