\section{Teorijske osnove i srodni radovi}
\label{chap:teorija}

Ovo poglavlje pruža pregled temeljnih teorijskih koncepata ključnih za razumijevanje predloženog modela optimizacije projektnih aktivnosti. Detaljno će se objasniti Problem ruksaka kao osnova za formulaciju problema raspodjele resursa, Genetski algoritmi kao optimizacijska metaheuristika, te Monte Carlo simulacija i metode procjene trajanja kao alati za modeliranje i analizu nesigurnosti u projektnom upravljanju.

\subsection{Problem odabira portfelja kao Knapsack problem}

Problem ruksaka (engl. \textit{Knapsack Problem}) jedan je od najpoznatijih problema kombinatorne optimizacije, svrstan u klasu NP-teških problema, što znači da ne postoji poznati algoritam koji bi ga mogao riješiti u polinomnom vremenu za sve slučajeve \cite{GareyJohnson1979, Kellerer2004}. U svojoj osnovnoj, jednodimenzionalnoj verziji (0/1 Knapsack Problem), cilj je odabrati podskup objekata iz danog skupa, od kojih svaki ima definiranu težinu i vrijednost. Optimizacijski zadatak je maksimizirati ukupnu vrijednost odabranih objekata, pod uvjetom da njihova ukupna težina ne prelazi unaprijed zadani kapacitet "ruksaka". Formalno, za skup od n objekata, gdje svaki objekt i ima težinu $w_i$ i vrijednost $v_i$, te uz zadani kapacitet ruksaka W, cilj je maksimizirati funkciju:
$$
\max \sum_{i=1}^n v_i x_i \quad \text{uz ograničenje} \quad \sum_{i=1}^n w_i x_i \leq W, x_i \in \{0,1\}
$$
gdje je $x_i=1$ ako je objekt i odabran, a $x_i=0$ ako nije.


Ova elegantna formulacija čini ga moćnim alatom za modeliranje problema alokacije resursa u stvarnom svijetu. U kontekstu upravljanja projektima, problem odabira portfelja aktivnosti prirodno poprima oblik složenije, višedimenzionalne varijante poznate kao \textit{Multi-Dimensional Knapsack Problem} (MDKP). U ovom modelu, projektne aktivnosti predstavljaju "objekte" koje želimo staviti u ruksak, a njihova vrijednost je povrat na investiciju (ROI) ili neki drugi pokazatelj strateške važnosti. "Težina" aktivnosti nije više jedna dimenzija, već vektor koji opisuje potrošnju različitih resursa – budžeta, radnih sati, specifične opreme, itd. Kapacitet ruksaka tada odgovara ukupnoj raspoloživosti svakog od tih resursa. MDKP stoga pruža robustan matematički okvir za rješavanje središnjeg problema ovog rada: kako optimalno raspodijeliti ograničene, višestruke resurse među konkurentskim projektnim aktivnostima.

\subsection{Genetski algoritmi kao metaheuristika za pretraživanje}

Genetski algoritmi (GA) su moćne metaheurističke optimizacijske metode inspirirane procesima prirodne selekcije i evolucije \cite{Holland1975, Goldberg1989}. Pripadaju široj klasi evolucijskih algoritama i iznimno su učinkoviti u rješavanju složenih optimizacijskih problema s velikim, diskretnim ili nelinearnim prostorima rješenja, posebno onih NP-teških, za koje klasične metode nisu praktične \cite{Gandomi2013, Mitchell1998}.

Umjesto da pretražuju prostor rješenja s jednom točkom, genetski algoritmi operiraju nad cijelom populacijom potencijalnih rješenja (kromosoma). Taj proces započinje generiranjem početnog, najčešće nasumičnog skupa rješenja. Zatim ulazi u iterativnu petlju u kojoj se simulira evolucija kroz generacije. U svakoj generaciji, svakom se rješenju (jedinki) dodjeljuje vrijednost pogodnosti (fitness) koja kvantificira njegovu kvalitetu prema definiranoj ciljnoj funkciji. Ovisno o cilju, funkcija pogodnosti može biti jedno-kriterijska (npr. maksimizacija ROI-a) ili, kao u naprednijem modelu ovog rada, više-kriterijska, gdje se istovremeno vrednuje više, često suprotstavljenih, ciljeva. Nakon evaluacije, provodi se selekcija, gdje jedinke s boljim fitnessom imaju veću vjerojatnost da budu odabrane kao "roditelji". Ti roditelji se zatim rekombiniraju pomoću operatora križanja (crossover), stvarajući "potomke" koji nasljeđuju i kombiniraju njihove karakteristike. Kako bi se održala raznolikost u populaciji i izbjeglo zaglavljivanje u lokalnim optimumima, na potomke se primjenjuje operator mutacije, koji unosi male, nasumične promjene. Na kraju, nova generacija potomaka zamjenjuje staru, i cijeli se proces ponavlja dok se ne zadovolji unaprijed definirani kriterij zaustavljanja, poput maksimalnog broja generacija.

\subsubsection{Više-kriterijski genetski algoritmi i NSGA-II}

Standardni genetski algoritmi su dizajnirani za probleme s jednim ciljem (jedno-kriterijska optimizacija), gdje je lako usporediti dva rješenja na temelju jedne vrijednosti pogodnosti. Međutim, mnogi stvarni problemi, uključujući i odabir projektnog portfelja, inherentno su više-kriterijski -- zahtijevaju istovremeno zadovoljenje više, često suprotstavljenih, ciljeva (npr. maksimizirati profit, a minimizirati rizik).

Za rješavanje takvih problema razvijeni su više-kriterijski evolucijski algoritmi (engl. \textit{Multi-Objective Evolutionary Algorithms}, MOEA). Njihov cilj nije pronaći jedno jedino "najbolje" rješenje, već skup rješenja koji predstavlja optimalan kompromis, poznat kao **Paretov front**. Ključni koncepti na kojima se temelje su:

\begin{itemize}
    \item \textbf{Dominacija:} Rješenje A "dominira" rješenje B ako je bolje ili jednako od B po svim kriterijima, a strogo bolje po barem jednom kriteriju.
    \item \textbf{Paretov front:} Skup svih rješenja u populaciji koja nisu dominirana ni od jednog drugog rješenja. To su najbolja kompromisna rješenja koja algoritam može ponuditi.
\end{itemize}

Jedan od najpoznatijih i najčešće korištenih algoritama u ovoj domeni je **NSGA-II** (\textit{Nondominated Sorting Genetic Algorithm II}) \cite{Deb2002}. Njegova efikasnost leži u dva ključna mehanizma:

\begin{enumerate}
    \item \textbf{Brzo sortiranje po nedominaciji (Fast non-dominated sorting):} U svakoj generaciji, algoritam rangira populaciju u slojeve (frontove) na temelju dominacije. Rješenja u prvom frontu su najbolja, zatim slijede rješenja u drugom frontu, i tako dalje. Ovim se osigurava da elitna, ne-dominirana rješenja imaju prednost pri selekciji.
    
    \item \textbf{Procjena gustoće naseljenosti (Crowding distance):} Kako bi se unutar istog fronta osigurala raznolikost i spriječilo grupiranje rješenja u samo jednom dijelu prostora, NSGA-II koristi mjeru "napučenosti". Prednost se daje rješenjima koja se nalaze u rjeđe naseljenim dijelovima Paretovog fronta, čime se dobiva širok i dobro raspoređen skup kompromisnih opcija.
\end{enumerate}

Zbog svoje robusnosti i efikasnosti u pronalaženju dobro raspoređenog Paretovog fronta, NSGA-II je odabran kao temelj za više-kriterijski hibridni model u ovom radu.
\subsection{Kvantifikacija rizika pomoću Monte Carlo simulacije}

Monte Carlo simulacija (MCS) je računska metoda koja koristi nasumično uzorkovanje za numeričko modeliranje i analizu složenih, stohastičkih sustava \cite{Rubinstein2016}. Njena primarna prednost leži u sposobnosti da pruži uvid u distribuciju mogućih ishoda nekog procesa, umjesto da daje samo jednu determinističku procjenu, što je čini nezamjenjivim alatom za kvantitativnu analizu rizika \cite{Vose2008}. U projektnom upravljanju, MCS se široko primjenjuje za procjenu vjerojatnih ishoda trajanja i troškova projekata, uzimajući u obzir inherentnu varijabilnost svake pojedine aktivnosti \cite{Miller2009, Avlijas2008}.

Proces Monte Carlo simulacije odvija se u nekoliko koraka. Prvo se identificiraju ključne ulazne varijable čije su vrijednosti neizvjesne, poput trajanja pojedinih aktivnosti. Za svaku takvu varijablu odabire se distribucija vjerojatnosti koja najbolje opisuje raspon i vjerojatnost njenih mogućih vrijednosti. Zatim, algoritam iterativno provodi tisuće "eksperimenata": u svakoj iteraciji, za svaku nesigurnu varijablu generira se jedna nasumična vrijednost iz njene definirane distribucije, te se na temelju tih vrijednosti izračunava ishod modela (npr. ukupno trajanje projekta). Ponavljanjem ovog procesa velik broj puta, dobiva se empirijska distribucija mogućih ishoda, iz koje se mogu iščitati ključni statistički pokazatelji poput očekivane vrijednosti, standardne devijacije i vjerojatnosti prekoračenja određenih pragova.

\subsection{Modeliranje nesigurnosti trajanja pomoću trotočkovne procjene}

Metodologija PERT (\textit{Program Evaluation and Review Technique}) uvela je u praksu korištenje tri vremenske procjene za aktivnosti s neizvjesnim trajanjem, što je danas standard u upravljanju projektnim rizikom \cite{Malcolm1959}. Ove tri točke su: $T_o$(optimistična procjena), $T_m$(najvjerojatnija procjena) i $T_p$(pesimistična procjena).

Dok tradicionalna PERT metoda koristi ove tri točke za izračun parametara Beta distribucije, u modernoj praksi, a posebno u Monte Carlo simulacijama, često se koristi Trokutasta distribucija zbog svoje jednostavnosti i intuitivnosti \cite{Law2015}. 

Trokutasta distribucija je kontinuirana distribucija vjerojatnosti definirana s tri parametra: minimum ($a$), maksimum ($b$) i najvjerojatnija vrijednost ($c$), što direktno odgovara procjenama $T_o$, $T_p$ i $T_m$. Njena je glavna prednost što ne zahtijeva opsežne povijesne podatke, već se može temeljiti na stručnom iskustvu, što je čini iznimno pogodnom za projektno planiranje. Slučajne vrijednosti generirane iz ove distribucije nalaze se unutar intervala [$T_o$, $T_p$], s najvećom vjerojatnošću pojavljivanja oko vrijednosti $T_m$. Prosječna vrijednost (očekivano trajanje) za Trokutastu distribuciju računa se jednostavnom formulom:

$$
E(T) = \frac{T_o + T_m + T_p}{3}
$$
Upravo je Trokutasta distribucija, zbog navedenih prednosti, odabrana kao temelj za modeliranje nesigurnosti trajanja aktivnosti u Monte Carlo simulacijama provedenim u ovom radu.
