\selectlanguage{croatian}
\begin{abstract}
Upravljanje projektima često uključuje složene odluke vezane uz raspodjelu aktivnosti i resursa, osobito u uvjetima nesigurnosti i vremenskih ograničenja. Tradicionalne metode kao što su PERT i CPM često ne uspijevaju obuhvatiti stohastičku prirodu stvarnih projekata. U ovom radu razvijen je i evaluiran hibridni optimizacijski pristup temeljen na genetskim algoritmima (GA) i Monte Carlo (MC) simulaciji. Kroz dvo-fazni eksperimentalni dizajn, provedena je sustavna usporedba tri modela: nasumične pretrage, jedno-kriterijskog GA usmjerenog isključivo na povrat na investiciju (ROI), te više-kriterijskog GA+MC modela (NSGA-II) koji istovremeno optimizira ROI i rizik trajanja projekta. Rezultati dobiveni na sintetičkim podacima različite složenosti i restriktivnosti pokazuju da, iako jedno-kriterijski GA najučinkovitije maksimizira profit, hibridni GA+MC model uspješno generira Paretov front rješenja koja nude optimalan kompromis između profitabilnosti i trajanja. Nadalje, istraživanje je otkrilo ključan nalaz: pod ekstremno restriktivnim budžetom, robusnost jednostavnijeg, jedno-kriterijskog GA nadmašuje onu složenijeg, više-kriterijskog modela. Rad zaključuje da ne postoji univerzalno superioran model, već da optimalan izbor ovisi o strateškim prioritetima – maksimizaciji profita naspram uravnoteženog upravljanja rizikom.
\end{abstract}
\begin{small}
\textbf{Ključne riječi:} projektno upravljanje, genetski algoritam, Monte Carlo simulacija, više-kriterijska optimizacija, upravljanje rizikom, Paretov front, NSGA-II
\end{small}

\bigskip

\selectlanguage{english}
\begin{abstract}
Project management often involves complex decisions regarding the allocation of activities and resources, especially under conditions of uncertainty and constraints. Traditional methods such as PERT and CPM frequently fail to capture the stochastic nature of real-world projects. This thesis develops and evaluates a hybrid optimization approach based on genetic algorithms (GA) and Monte Carlo (MC) simulation. Through a two-phase experimental design, a systematic comparison of three models was conducted: random search, a single-objective GA focused solely on return on investment (ROI), and a multi-objective GA+MC model (NSGA-II) that simultaneously optimizes ROI and project duration risk. The results, obtained from synthetic data of varying complexity and restrictiveness, show that while the single-objective GA is most effective at maximizing profit, the hybrid GA+MC model successfully generates a Pareto front of solutions offering an optimal trade-off between profitability and duration. Furthermore, the research revealed a key finding: under extremely restrictive budget constraints, the robustness of the simpler, single-objective GA surpasses that of the more complex, multi-objective model. The thesis concludes that there is no universally superior model; rather, the optimal choice depends on strategic priorities—maximizing profit versus balanced risk management.
\end{abstract}
\begin{small}
\textbf{Keywords:} project management, genetic algorithm, Monte Carlo simulation, multi-objective optimization, risk management, Pareto front, NSGA-II
\end{small}
