\selectlanguage{croatian}
\begin{abstract}
Upravljanje projektima često uključuje složene odluke vezane uz raspodjelu aktivnosti i resursa, osobito u uvjetima nesigurnosti i vremenskih ograničenja. Tradicionalne metode kao što su PERT i CPM često ne uspijevaju obuhvatiti stohastičku prirodu stvarnih projekata. U ovom radu predlaže se model koji kombinira genetske algoritme i Monte Carlo simulaciju s ciljem postizanja robusne optimizacije raspodjele projektnih aktivnosti. Genetski algoritam koristi se za pretraživanje prostora mogućih rješenja, dok Monte Carlo simulacija omogućava procjenu utjecaja varijabilnih trajanja aktivnosti na ukupno trajanje projekta. Eksperimentalna evaluacija modela provodi se na simuliranim projektnim podacima, a rezultati pokazuju poboljšanu robusnost i učinkovitost u odnosu na determinističke pristupe.
\end{abstract}
\begin{small}
\textbf{Ključne riječi} : projektno upravljanje, genetski algoritam, Monte Carlo simulacija, optimizacija rasporeda, raspodjela aktivnosti
\end{small}

\bigskip

\selectlanguage{english}
\begin{abstract}

Project management often involves complex decisions related to the allocation of activities and resources, especially under uncertainty and time constraints. Traditional methods such as PERT and CPM frequently fail to capture the stochastic nature of real-world projects. This thesis proposes a model that combines genetic algorithms and Monte Carlo simulation to achieve robust optimization of project activity allocation. The genetic algorithm is used to explore the space of possible solutions, while Monte Carlo simulation estimates the impact of variable activity durations on the overall project timeline. The model is experimentally evaluated using simulated project data, and results indicate improved robustness and efficiency compared to deterministic approaches.
\end{abstract}
\begin{small}
\textbf{Keywords} : project management, genetic algorithm, Monte Carlo simulation, schedule optimization, activity allocation
\end{small}
