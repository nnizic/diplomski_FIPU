
\section{Implementacija}

Razvijeni model optimizacije implementiran je u programskom jeziku \textbf{Python}, odabranom zbog čitljivosti, bogatog ekosustava biblioteka i široke primjene u znanstvenom računarstvu \cite{PythonSoftwareFoundation}. Python omogućuje brzu izradu prototipa, jednostavnu integraciju modula te učinkovitu obradu i vizualizaciju podataka.

\subsection{Korištene biblioteke}

Za izradu sustava korištene su sljedeće biblioteke (Tablica~\ref{tab:biblioteke}):

\begin{table}[H]
\centering
\caption{Korištene biblioteke u implementaciji}
\label{tab:biblioteke}
\begin{tabular}{|l|p{10cm}|}
\hline
\textbf{Biblioteka} & \textbf{Namjena} \\ \hline
NumPy & Numeričke operacije, generiranje slučajnih brojeva, vektorizacija izračuna. \\ \hline
SciPy & Statističke distribucije i znanstveno računarstvo; korišten za modeliranje PERT distribucije. \\ \hline
Matplotlib & Vizualizacija rezultata simulacija i optimizacijskih procesa. \\ \hline
DEAP & Implementacija genetskog algoritma, definiranje operatora selekcije, križanja i mutacije. \\ \hline
\end{tabular}
\end{table}

\subsection{Struktura sustava}

Sustav se sastoji od tri modula:
\begin{enumerate}
    \item \textbf{Monte Carlo modul} – generira distribucije trajanja zadataka i procjenjuje nesigurnosti.
    \item \textbf{Genetski algoritam modul} – optimizira raspodjelu aktivnosti na temelju rezultata Monte Carlo simulacija.
    \item \textbf{Modul za vizualizaciju} – grafički prikazuje distribucije, rezultate i tijek optimizacije.
\end{enumerate}

\subsection{Monte Carlo simulacija}

Za svaki zadatak definirane su tri procjene trajanja: optimistična $(a)$, najvjerojatnija $(m)$ i pesimistična $(b)$.  
Distribucija trajanja modelirana je pomoću \textit{PERT distribucije} (Beta-PERT), gdje se očekivana vrijednost računa kao:

\[
\mu = \frac{a + 4m + b}{6}
\]

Parametri $\alpha_1$ i $\alpha_2$ Beta distribucije izračunavaju se prema \cite{Vose2008}:

\[
\alpha_1 = \frac{(\mu - a) \cdot (2m - a - b)}{(m - \mu)(b - a)}, \quad
\alpha_2 = \frac{\alpha_1 \cdot (b - \mu)}{(\mu - a)}
\]

Ako $\alpha_1 \leq 0$ ili $\alpha_2 \leq 0$, vrijednost se postavlja na $1.0$ radi numeričke stabilnosti.

\subsection{Genetski algoritam}

Implementacija genetskog algoritma provedena je pomoću biblioteke DEAP \cite{DEAP2012}.  
Svaki kromosom predstavlja potencijalni raspored aktivnosti, a optimizacija se temelji na višekriterijskoj funkciji pogodnosti koja uključuje:
\begin{itemize}
    \item trajanje projekta (na temelju Monte Carlo simulacije),
    \item troškove,
    \item dodatne projektne parametre.
\end{itemize}

Korišteni su sljedeći operatori:
\begin{itemize}
    \item \textbf{Selekcija:} turnirska i rulet-kolo selekcija,
    \item \textbf{Križanje:} PMX i OX1,
    \item \textbf{Mutacija:} zamjena pozicija gena (swap) i umetanje (insert).
\end{itemize}

\subsection{Vizualizacija}

Rezultati simulacija i optimizacije vizualizirani su pomoću \textbf{Matplotlib} \cite{Hunter2007} biblioteke kroz:
\begin{itemize}
    \item histograme distribucija trajanja zadataka,
    \item grafove konvergencije genetskog algoritma,
    \item usporedbe optimiziranih rješenja.
\end{itemize}


